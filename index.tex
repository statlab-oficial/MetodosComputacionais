% Options for packages loaded elsewhere
\PassOptionsToPackage{unicode}{hyperref}
\PassOptionsToPackage{hyphens}{url}
\PassOptionsToPackage{dvipsnames,svgnames,x11names}{xcolor}
%
\documentclass[
  letterpaper,
  DIV=11,
  numbers=noendperiod]{scrreprt}

\usepackage{amsmath,amssymb}
\usepackage{iftex}
\ifPDFTeX
  \usepackage[T1]{fontenc}
  \usepackage[utf8]{inputenc}
  \usepackage{textcomp} % provide euro and other symbols
\else % if luatex or xetex
  \usepackage{unicode-math}
  \defaultfontfeatures{Scale=MatchLowercase}
  \defaultfontfeatures[\rmfamily]{Ligatures=TeX,Scale=1}
\fi
\usepackage{lmodern}
\ifPDFTeX\else  
    % xetex/luatex font selection
\fi
% Use upquote if available, for straight quotes in verbatim environments
\IfFileExists{upquote.sty}{\usepackage{upquote}}{}
\IfFileExists{microtype.sty}{% use microtype if available
  \usepackage[]{microtype}
  \UseMicrotypeSet[protrusion]{basicmath} % disable protrusion for tt fonts
}{}
\makeatletter
\@ifundefined{KOMAClassName}{% if non-KOMA class
  \IfFileExists{parskip.sty}{%
    \usepackage{parskip}
  }{% else
    \setlength{\parindent}{0pt}
    \setlength{\parskip}{6pt plus 2pt minus 1pt}}
}{% if KOMA class
  \KOMAoptions{parskip=half}}
\makeatother
\usepackage{xcolor}
\setlength{\emergencystretch}{3em} % prevent overfull lines
\setcounter{secnumdepth}{5}
% Make \paragraph and \subparagraph free-standing
\ifx\paragraph\undefined\else
  \let\oldparagraph\paragraph
  \renewcommand{\paragraph}[1]{\oldparagraph{#1}\mbox{}}
\fi
\ifx\subparagraph\undefined\else
  \let\oldsubparagraph\subparagraph
  \renewcommand{\subparagraph}[1]{\oldsubparagraph{#1}\mbox{}}
\fi


\providecommand{\tightlist}{%
  \setlength{\itemsep}{0pt}\setlength{\parskip}{0pt}}\usepackage{longtable,booktabs,array}
\usepackage{calc} % for calculating minipage widths
% Correct order of tables after \paragraph or \subparagraph
\usepackage{etoolbox}
\makeatletter
\patchcmd\longtable{\par}{\if@noskipsec\mbox{}\fi\par}{}{}
\makeatother
% Allow footnotes in longtable head/foot
\IfFileExists{footnotehyper.sty}{\usepackage{footnotehyper}}{\usepackage{footnote}}
\makesavenoteenv{longtable}
\usepackage{graphicx}
\makeatletter
\def\maxwidth{\ifdim\Gin@nat@width>\linewidth\linewidth\else\Gin@nat@width\fi}
\def\maxheight{\ifdim\Gin@nat@height>\textheight\textheight\else\Gin@nat@height\fi}
\makeatother
% Scale images if necessary, so that they will not overflow the page
% margins by default, and it is still possible to overwrite the defaults
% using explicit options in \includegraphics[width, height, ...]{}
\setkeys{Gin}{width=\maxwidth,height=\maxheight,keepaspectratio}
% Set default figure placement to htbp
\makeatletter
\def\fps@figure{htbp}
\makeatother
% definitions for citeproc citations
\NewDocumentCommand\citeproctext{}{}
\NewDocumentCommand\citeproc{mm}{%
  \begingroup\def\citeproctext{#2}\cite{#1}\endgroup}
\makeatletter
 % allow citations to break across lines
 \let\@cite@ofmt\@firstofone
 % avoid brackets around text for \cite:
 \def\@biblabel#1{}
 \def\@cite#1#2{{#1\if@tempswa , #2\fi}}
\makeatother
\newlength{\cslhangindent}
\setlength{\cslhangindent}{1.5em}
\newlength{\csllabelwidth}
\setlength{\csllabelwidth}{3em}
\newenvironment{CSLReferences}[2] % #1 hanging-indent, #2 entry-spacing
 {\begin{list}{}{%
  \setlength{\itemindent}{0pt}
  \setlength{\leftmargin}{0pt}
  \setlength{\parsep}{0pt}
  % turn on hanging indent if param 1 is 1
  \ifodd #1
   \setlength{\leftmargin}{\cslhangindent}
   \setlength{\itemindent}{-1\cslhangindent}
  \fi
  % set entry spacing
  \setlength{\itemsep}{#2\baselineskip}}}
 {\end{list}}
\usepackage{calc}
\newcommand{\CSLBlock}[1]{\hfill\break\parbox[t]{\linewidth}{\strut\ignorespaces#1\strut}}
\newcommand{\CSLLeftMargin}[1]{\parbox[t]{\csllabelwidth}{\strut#1\strut}}
\newcommand{\CSLRightInline}[1]{\parbox[t]{\linewidth - \csllabelwidth}{\strut#1\strut}}
\newcommand{\CSLIndent}[1]{\hspace{\cslhangindent}#1}

\KOMAoption{captions}{tableheading}
\makeatletter
\@ifpackageloaded{bookmark}{}{\usepackage{bookmark}}
\makeatother
\makeatletter
\@ifpackageloaded{caption}{}{\usepackage{caption}}
\AtBeginDocument{%
\ifdefined\contentsname
  \renewcommand*\contentsname{Índice}
\else
  \newcommand\contentsname{Índice}
\fi
\ifdefined\listfigurename
  \renewcommand*\listfigurename{Lista de Figuras}
\else
  \newcommand\listfigurename{Lista de Figuras}
\fi
\ifdefined\listtablename
  \renewcommand*\listtablename{Lista de Tabelas}
\else
  \newcommand\listtablename{Lista de Tabelas}
\fi
\ifdefined\figurename
  \renewcommand*\figurename{Figura}
\else
  \newcommand\figurename{Figura}
\fi
\ifdefined\tablename
  \renewcommand*\tablename{Tabela}
\else
  \newcommand\tablename{Tabela}
\fi
}
\@ifpackageloaded{float}{}{\usepackage{float}}
\floatstyle{ruled}
\@ifundefined{c@chapter}{\newfloat{codelisting}{h}{lop}}{\newfloat{codelisting}{h}{lop}[chapter]}
\floatname{codelisting}{Listagem}
\newcommand*\listoflistings{\listof{codelisting}{Lista de Listagens}}
\makeatother
\makeatletter
\makeatother
\makeatletter
\@ifpackageloaded{caption}{}{\usepackage{caption}}
\@ifpackageloaded{subcaption}{}{\usepackage{subcaption}}
\makeatother
\ifLuaTeX
\usepackage[bidi=basic]{babel}
\else
\usepackage[bidi=default]{babel}
\fi
\babelprovide[main,import]{brazilian}
% get rid of language-specific shorthands (see #6817):
\let\LanguageShortHands\languageshorthands
\def\languageshorthands#1{}
\ifLuaTeX
  \usepackage{selnolig}  % disable illegal ligatures
\fi
\usepackage{bookmark}

\IfFileExists{xurl.sty}{\usepackage{xurl}}{} % add URL line breaks if available
\urlstyle{same} % disable monospaced font for URLs
\hypersetup{
  pdftitle={Métodos Computacionais},
  pdfauthor={Ronald Targino, Rafael Braz, Juvêncio Nobre e Manoel Santos-Neto},
  pdflang={pt-BR},
  colorlinks=true,
  linkcolor={blue},
  filecolor={Maroon},
  citecolor={Blue},
  urlcolor={Blue},
  pdfcreator={LaTeX via pandoc}}

\title{Métodos Computacionais}
\usepackage{etoolbox}
\makeatletter
\providecommand{\subtitle}[1]{% add subtitle to \maketitle
  \apptocmd{\@title}{\par {\large #1 \par}}{}{}
}
\makeatother
\subtitle{Departamento de Estatística e Matemática Aplicada}
\author{Ronald Targino, Rafael Braz, Juvêncio Nobre e Manoel
Santos-Neto}
\date{2026-03-08}

\begin{document}
\maketitle

\renewcommand*\contentsname{Índice}
{
\hypersetup{linkcolor=}
\setcounter{tocdepth}{2}
\tableofcontents
}
\bookmarksetup{startatroot}

\chapter*{Prefácio}\label{prefuxe1cio}
\addcontentsline{toc}{chapter}{Prefácio}

\markboth{Prefácio}{Prefácio}

Este livro resulta de anos de experiência em sala de aula dos
professores Ronald Targino, Rafael Braz, Juvêncio Nobre e Manoel
Santos-Neto. Destina-se a apoiar os alunos da graduação em Estatística e
do Programa de Pós-Graduação em Modelagem e Métodos Quantitativos
(PPGMMQ) do Departamento de Estatística e Matemática Aplicada (DEMA) da
Universidade Federal do Ceará (UFC).

Ao longo dos capítulos, abordamos a geração de números aleatórios
(discretos e contínuos); métodos de suavização; simulação estocástica
por inversão, rejeição e composição, bem como métodos de reamostragem;
métodos de aproximação e integração; quadratura Gaussiana, integração de
Monte Carlo e quadratura adaptativa; métodos de Monte Carlo em sentido
amplo; amostradores MCMC, com ênfase em Gibbs e Metropolis--Hastings;
otimização numérica via Newton--Raphson, Fisher scoring e quase-Newton,
além do algoritmo EM; Bootstrap e Jackknife; diagnóstico de
convergência; e aspectos computacionais em problemas práticos, com foco
em implementação eficiente, estabilidade numérica e reprodutibilidade
dos resultados.

Esperamos que este material sirva não apenas como texto-base para as
disciplinas Estatística Computacional (graduação em Estatística) e
Métodos Computacionais em Estatística (Mestrado-PPGMMQ), mas também como
suporte para aqueles que desejam programar com qualidade nas áreas de
Estatística e Ciência de Dados.

\bookmarksetup{startatroot}

\chapter{Introdução}\label{introduuxe7uxe3o}

\bookmarksetup{startatroot}

\chapter{Motivação}\label{motivauxe7uxe3o}

Os estudantes adoram jogos e a descoberta prática, e a simulação
facilita o engajamento nessas atividades, ao mesmo tempo que ilustra
resultados que podem ser não intuitivos, bem como teoria geral, como a
Lei dos Grandes Números.

A simulação tem um papel preponderante na estatística moderna, e suas
vantagens no ensino de estatística são conhecidas há muito tempo. Em um
de seus primeiros números, este periódico publicou artigos que aludem
precisamente a isso. Thomas e Moore (1980) afirmaram que ``a introdução
do computador na sala de aula escolar trouxe uma nova técnica para o
ensino, a técnica da simulação''. Zieffler e Garfield (2007) e Tintle et
al.~(2015) discutem o papel e a importância da aprendizagem baseada em
simulação no currículo de graduação em estatística. No entanto, outros
autores (por exemplo, Hodgson e Burke, 2000) discutem alguns problemas
que podem surgir ao ensinar uma disciplina por meio de simulação, a
saber, o desenvolvimento de certos equívocos na mente dos estudantes. A
atividade aqui discutida é o conhecido e amplamente divulgado problema
do aniversário (ver, por exemplo, Falk, 2014). Aqui, seguimos o exemplo
de Matthews e Stones (1998), considerando duas equipes de futebol e,
portanto, coincidências de aniversário entre 22 jogadores. Um resultado
positivo importante dessa atividade é a discussão que surgirá
naturalmente entre os estudantes, com o professor atuando como mediador.

Para iniciar a discussão, propõe-se o seguinte problema:

\textbf{O problema:} Em uma partida de futebol, qual é a probabilidade
de que pelo menos dois dos 22 jogadores façam aniversário no mesmo dia?

Em um pais chamado de país do futebol, o contexto é proposital: o
futebol é popular e as probabilidades resultantes são contraintuitivas.
Antes de qualquer cálculo, considere as hipóteses: (i) todos os 365 dias
do ano são igualmente prováveis para qualquer aniversário; (ii) as datas
de aniversário dos jogadores são independentes entre si.

\section{Atividade: Problema do Aniversário (22
jogadores)}\label{atividade-problema-do-aniversuxe1rio-22-jogadores}

\textbf{Objetivos}\\
- Estimar, via simulação, a probabilidade de coincidência de
aniversários.\\
- Relacionar frequência relativa, LGN e variação amostral.

\textbf{Hipóteses}\\
- 365 dias equiprováveis; datas independentes; ignorar bissexto/gêmeos.

\textbf{Materiais}\\
- R (ou Posit Cloud), roteiro com comandos \texttt{sample()},
\texttt{table()}, \texttt{mean()}.

\bookmarksetup{startatroot}

\chapter{Números Pseudoaleatórios}\label{nuxfameros-pseudoaleatuxf3rios}

\bookmarksetup{startatroot}

\chapter{Otimização Numérica}\label{otimizauxe7uxe3o-numuxe9rica}

\bookmarksetup{startatroot}

\chapter{Métodos de Reamostragem}\label{muxe9todos-de-reamostragem}

\bookmarksetup{startatroot}

\chapter{Métodos de Monte Carlo}\label{muxe9todos-de-monte-carlo}

\bookmarksetup{startatroot}

\chapter{Algoritmo EM}\label{algoritmo-em}

\bookmarksetup{startatroot}

\chapter{Métodos Adicionais}\label{muxe9todos-adicionais}

\bookmarksetup{startatroot}

\chapter*{References}\label{references}
\addcontentsline{toc}{chapter}{References}

\markboth{References}{References}

\phantomsection\label{refs}
\begin{CSLReferences}{0}{1}
\end{CSLReferences}



\end{document}
