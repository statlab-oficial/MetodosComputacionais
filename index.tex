% Options for packages loaded elsewhere
\PassOptionsToPackage{unicode}{hyperref}
\PassOptionsToPackage{hyphens}{url}
\PassOptionsToPackage{dvipsnames,svgnames,x11names}{xcolor}
%
\documentclass[
  letterpaper,
  DIV=11,
  numbers=noendperiod]{scrreprt}

\usepackage{amsmath,amssymb}
\usepackage{iftex}
\ifPDFTeX
  \usepackage[T1]{fontenc}
  \usepackage[utf8]{inputenc}
  \usepackage{textcomp} % provide euro and other symbols
\else % if luatex or xetex
  \usepackage{unicode-math}
  \defaultfontfeatures{Scale=MatchLowercase}
  \defaultfontfeatures[\rmfamily]{Ligatures=TeX,Scale=1}
\fi
\usepackage{lmodern}
\ifPDFTeX\else  
    % xetex/luatex font selection
\fi
% Use upquote if available, for straight quotes in verbatim environments
\IfFileExists{upquote.sty}{\usepackage{upquote}}{}
\IfFileExists{microtype.sty}{% use microtype if available
  \usepackage[]{microtype}
  \UseMicrotypeSet[protrusion]{basicmath} % disable protrusion for tt fonts
}{}
\makeatletter
\@ifundefined{KOMAClassName}{% if non-KOMA class
  \IfFileExists{parskip.sty}{%
    \usepackage{parskip}
  }{% else
    \setlength{\parindent}{0pt}
    \setlength{\parskip}{6pt plus 2pt minus 1pt}}
}{% if KOMA class
  \KOMAoptions{parskip=half}}
\makeatother
\usepackage{xcolor}
\setlength{\emergencystretch}{3em} % prevent overfull lines
\setcounter{secnumdepth}{5}
% Make \paragraph and \subparagraph free-standing
\ifx\paragraph\undefined\else
  \let\oldparagraph\paragraph
  \renewcommand{\paragraph}[1]{\oldparagraph{#1}\mbox{}}
\fi
\ifx\subparagraph\undefined\else
  \let\oldsubparagraph\subparagraph
  \renewcommand{\subparagraph}[1]{\oldsubparagraph{#1}\mbox{}}
\fi

\usepackage{color}
\usepackage{fancyvrb}
\newcommand{\VerbBar}{|}
\newcommand{\VERB}{\Verb[commandchars=\\\{\}]}
\DefineVerbatimEnvironment{Highlighting}{Verbatim}{commandchars=\\\{\}}
% Add ',fontsize=\small' for more characters per line
\usepackage{framed}
\definecolor{shadecolor}{RGB}{241,243,245}
\newenvironment{Shaded}{\begin{snugshade}}{\end{snugshade}}
\newcommand{\AlertTok}[1]{\textcolor[rgb]{0.68,0.00,0.00}{#1}}
\newcommand{\AnnotationTok}[1]{\textcolor[rgb]{0.37,0.37,0.37}{#1}}
\newcommand{\AttributeTok}[1]{\textcolor[rgb]{0.40,0.45,0.13}{#1}}
\newcommand{\BaseNTok}[1]{\textcolor[rgb]{0.68,0.00,0.00}{#1}}
\newcommand{\BuiltInTok}[1]{\textcolor[rgb]{0.00,0.23,0.31}{#1}}
\newcommand{\CharTok}[1]{\textcolor[rgb]{0.13,0.47,0.30}{#1}}
\newcommand{\CommentTok}[1]{\textcolor[rgb]{0.37,0.37,0.37}{#1}}
\newcommand{\CommentVarTok}[1]{\textcolor[rgb]{0.37,0.37,0.37}{\textit{#1}}}
\newcommand{\ConstantTok}[1]{\textcolor[rgb]{0.56,0.35,0.01}{#1}}
\newcommand{\ControlFlowTok}[1]{\textcolor[rgb]{0.00,0.23,0.31}{#1}}
\newcommand{\DataTypeTok}[1]{\textcolor[rgb]{0.68,0.00,0.00}{#1}}
\newcommand{\DecValTok}[1]{\textcolor[rgb]{0.68,0.00,0.00}{#1}}
\newcommand{\DocumentationTok}[1]{\textcolor[rgb]{0.37,0.37,0.37}{\textit{#1}}}
\newcommand{\ErrorTok}[1]{\textcolor[rgb]{0.68,0.00,0.00}{#1}}
\newcommand{\ExtensionTok}[1]{\textcolor[rgb]{0.00,0.23,0.31}{#1}}
\newcommand{\FloatTok}[1]{\textcolor[rgb]{0.68,0.00,0.00}{#1}}
\newcommand{\FunctionTok}[1]{\textcolor[rgb]{0.28,0.35,0.67}{#1}}
\newcommand{\ImportTok}[1]{\textcolor[rgb]{0.00,0.46,0.62}{#1}}
\newcommand{\InformationTok}[1]{\textcolor[rgb]{0.37,0.37,0.37}{#1}}
\newcommand{\KeywordTok}[1]{\textcolor[rgb]{0.00,0.23,0.31}{#1}}
\newcommand{\NormalTok}[1]{\textcolor[rgb]{0.00,0.23,0.31}{#1}}
\newcommand{\OperatorTok}[1]{\textcolor[rgb]{0.37,0.37,0.37}{#1}}
\newcommand{\OtherTok}[1]{\textcolor[rgb]{0.00,0.23,0.31}{#1}}
\newcommand{\PreprocessorTok}[1]{\textcolor[rgb]{0.68,0.00,0.00}{#1}}
\newcommand{\RegionMarkerTok}[1]{\textcolor[rgb]{0.00,0.23,0.31}{#1}}
\newcommand{\SpecialCharTok}[1]{\textcolor[rgb]{0.37,0.37,0.37}{#1}}
\newcommand{\SpecialStringTok}[1]{\textcolor[rgb]{0.13,0.47,0.30}{#1}}
\newcommand{\StringTok}[1]{\textcolor[rgb]{0.13,0.47,0.30}{#1}}
\newcommand{\VariableTok}[1]{\textcolor[rgb]{0.07,0.07,0.07}{#1}}
\newcommand{\VerbatimStringTok}[1]{\textcolor[rgb]{0.13,0.47,0.30}{#1}}
\newcommand{\WarningTok}[1]{\textcolor[rgb]{0.37,0.37,0.37}{\textit{#1}}}

\providecommand{\tightlist}{%
  \setlength{\itemsep}{0pt}\setlength{\parskip}{0pt}}\usepackage{longtable,booktabs,array}
\usepackage{calc} % for calculating minipage widths
% Correct order of tables after \paragraph or \subparagraph
\usepackage{etoolbox}
\makeatletter
\patchcmd\longtable{\par}{\if@noskipsec\mbox{}\fi\par}{}{}
\makeatother
% Allow footnotes in longtable head/foot
\IfFileExists{footnotehyper.sty}{\usepackage{footnotehyper}}{\usepackage{footnote}}
\makesavenoteenv{longtable}
\usepackage{graphicx}
\makeatletter
\def\maxwidth{\ifdim\Gin@nat@width>\linewidth\linewidth\else\Gin@nat@width\fi}
\def\maxheight{\ifdim\Gin@nat@height>\textheight\textheight\else\Gin@nat@height\fi}
\makeatother
% Scale images if necessary, so that they will not overflow the page
% margins by default, and it is still possible to overwrite the defaults
% using explicit options in \includegraphics[width, height, ...]{}
\setkeys{Gin}{width=\maxwidth,height=\maxheight,keepaspectratio}
% Set default figure placement to htbp
\makeatletter
\def\fps@figure{htbp}
\makeatother
% definitions for citeproc citations
\NewDocumentCommand\citeproctext{}{}
\NewDocumentCommand\citeproc{mm}{%
  \begingroup\def\citeproctext{#2}\cite{#1}\endgroup}
\makeatletter
 % allow citations to break across lines
 \let\@cite@ofmt\@firstofone
 % avoid brackets around text for \cite:
 \def\@biblabel#1{}
 \def\@cite#1#2{{#1\if@tempswa , #2\fi}}
\makeatother
\newlength{\cslhangindent}
\setlength{\cslhangindent}{1.5em}
\newlength{\csllabelwidth}
\setlength{\csllabelwidth}{3em}
\newenvironment{CSLReferences}[2] % #1 hanging-indent, #2 entry-spacing
 {\begin{list}{}{%
  \setlength{\itemindent}{0pt}
  \setlength{\leftmargin}{0pt}
  \setlength{\parsep}{0pt}
  % turn on hanging indent if param 1 is 1
  \ifodd #1
   \setlength{\leftmargin}{\cslhangindent}
   \setlength{\itemindent}{-1\cslhangindent}
  \fi
  % set entry spacing
  \setlength{\itemsep}{#2\baselineskip}}}
 {\end{list}}
\usepackage{calc}
\newcommand{\CSLBlock}[1]{\hfill\break\parbox[t]{\linewidth}{\strut\ignorespaces#1\strut}}
\newcommand{\CSLLeftMargin}[1]{\parbox[t]{\csllabelwidth}{\strut#1\strut}}
\newcommand{\CSLRightInline}[1]{\parbox[t]{\linewidth - \csllabelwidth}{\strut#1\strut}}
\newcommand{\CSLIndent}[1]{\hspace{\cslhangindent}#1}

% pacotes úteis
\usepackage{amsmath}
\usepackage{array}      % para o \begin{array} usado pelo \stackunder
% \usepackage{stackengine} % (opcional) se quiser usar a versão do pacote \stackunder

% suas macros
\newcommand{\tn}{\mbox{\boldmath $\theta$} }
\newcommand{\sn}{\mbox{\boldmath $\sigma$} }
\newcommand{\On}{\mbox{\boldmath $\Omega$} }
\newcommand{\Dn}{\mbox{\boldmath $\Delta$} }
\newcommand{\In}{\mbox{\textbf{I}} }
\newcommand{\yn}{\mbox{\textbf{y}} }
\newcommand{\Vn}{\mbox{\textbf{V}} }
\newcommand{\mn}{\mbox{\boldmath $\mu$} }
\newcommand{\Fn}{\mbox{\boldmath $\Phi$} }
\newcommand{\Pn}{\mbox{\boldmath $\Pi$} }

\newenvironment{Dproof}[1][DProof]{\textbf{#1.} }{\ \rule{0.2cm}{0.2cm}} 
\newcommand{\stackunder}[2]{ \renewcommand{\arraystretch}{0.2} 
\displaystyle \begin{array}[t]{c}  {#1}\\_{#2}\end{array} 
\renewcommand{\arraystretch}{1}} 
\newcommand{\vect}[1]{\!\!\!\stackunder{#1}{\sim}\!\!\!}
\renewcommand{\Re}{I \! \! R}
\newcommand{\real}{\mbox{$I\!\!R$}}
\newcommand{\bi}{\begin{itemize}}
\newcommand{\ei}{\end{itemize}}
\newcommand{\va}{$X_1,\,X_2,\,\dots,\,X_n$}
\newcommand{\vam}{$x_1,\,x_2\,\ldots,\,x_n$}
\KOMAoption{captions}{tableheading}
\makeatletter
\@ifpackageloaded{tcolorbox}{}{\usepackage[skins,breakable]{tcolorbox}}
\@ifpackageloaded{fontawesome5}{}{\usepackage{fontawesome5}}
\definecolor{quarto-callout-color}{HTML}{909090}
\definecolor{quarto-callout-note-color}{HTML}{0758E5}
\definecolor{quarto-callout-important-color}{HTML}{CC1914}
\definecolor{quarto-callout-warning-color}{HTML}{EB9113}
\definecolor{quarto-callout-tip-color}{HTML}{00A047}
\definecolor{quarto-callout-caution-color}{HTML}{FC5300}
\definecolor{quarto-callout-color-frame}{HTML}{acacac}
\definecolor{quarto-callout-note-color-frame}{HTML}{4582ec}
\definecolor{quarto-callout-important-color-frame}{HTML}{d9534f}
\definecolor{quarto-callout-warning-color-frame}{HTML}{f0ad4e}
\definecolor{quarto-callout-tip-color-frame}{HTML}{02b875}
\definecolor{quarto-callout-caution-color-frame}{HTML}{fd7e14}
\makeatother
\makeatletter
\@ifpackageloaded{bookmark}{}{\usepackage{bookmark}}
\makeatother
\makeatletter
\@ifpackageloaded{caption}{}{\usepackage{caption}}
\AtBeginDocument{%
\ifdefined\contentsname
  \renewcommand*\contentsname{Índice}
\else
  \newcommand\contentsname{Índice}
\fi
\ifdefined\listfigurename
  \renewcommand*\listfigurename{Lista de Figuras}
\else
  \newcommand\listfigurename{Lista de Figuras}
\fi
\ifdefined\listtablename
  \renewcommand*\listtablename{Lista de Tabelas}
\else
  \newcommand\listtablename{Lista de Tabelas}
\fi
\ifdefined\figurename
  \renewcommand*\figurename{Figura}
\else
  \newcommand\figurename{Figura}
\fi
\ifdefined\tablename
  \renewcommand*\tablename{Tabela}
\else
  \newcommand\tablename{Tabela}
\fi
}
\@ifpackageloaded{float}{}{\usepackage{float}}
\floatstyle{ruled}
\@ifundefined{c@chapter}{\newfloat{codelisting}{h}{lop}}{\newfloat{codelisting}{h}{lop}[chapter]}
\floatname{codelisting}{Listagem}
\newcommand*\listoflistings{\listof{codelisting}{Lista de Listagens}}
\makeatother
\makeatletter
\makeatother
\makeatletter
\@ifpackageloaded{caption}{}{\usepackage{caption}}
\@ifpackageloaded{subcaption}{}{\usepackage{subcaption}}
\makeatother
\ifLuaTeX
\usepackage[bidi=basic]{babel}
\else
\usepackage[bidi=default]{babel}
\fi
\babelprovide[main,import]{brazilian}
% get rid of language-specific shorthands (see #6817):
\let\LanguageShortHands\languageshorthands
\def\languageshorthands#1{}
\ifLuaTeX
  \usepackage{selnolig}  % disable illegal ligatures
\fi
\usepackage{bookmark}

\IfFileExists{xurl.sty}{\usepackage{xurl}}{} % add URL line breaks if available
\urlstyle{same} % disable monospaced font for URLs
\hypersetup{
  pdftitle={Métodos Computacionais},
  pdfauthor={Ronald Targino, Rafael Braz, Juvêncio Nobre e Manoel Santos-Neto},
  pdflang={pt-BR},
  colorlinks=true,
  linkcolor={blue},
  filecolor={Maroon},
  citecolor={Blue},
  urlcolor={Blue},
  pdfcreator={LaTeX via pandoc}}

\title{Métodos Computacionais}
\usepackage{etoolbox}
\makeatletter
\providecommand{\subtitle}[1]{% add subtitle to \maketitle
  \apptocmd{\@title}{\par {\large #1 \par}}{}{}
}
\makeatother
\subtitle{Departamento de Estatística e Matemática Aplicada}
\author{Ronald Targino, Rafael Braz, Juvêncio Nobre e Manoel
Santos-Neto}
\date{2025-09-05}

\begin{document}
\maketitle

\renewcommand*\contentsname{Índice}
{
\hypersetup{linkcolor=}
\setcounter{tocdepth}{2}
\tableofcontents
}
\bookmarksetup{startatroot}

\chapter*{Prefácio}\label{prefuxe1cio}
\addcontentsline{toc}{chapter}{Prefácio}

\markboth{Prefácio}{Prefácio}

Este livro resulta de anos de experiência em sala de aula dos
professores Ronald Targino, Rafael Braz, Juvêncio Nobre e Manoel
Santos-Neto. Destina-se a apoiar os alunos da graduação em Estatística e
do Programa de Pós-Graduação em Modelagem e Métodos Quantitativos
(PPGMMQ) do Departamento de Estatística e Matemática Aplicada (DEMA) da
Universidade Federal do Ceará (UFC).

Ao longo dos capítulos, abordamos a geração de números aleatórios
(discretos e contínuos); métodos de suavização; simulação estocástica
por inversão, rejeição e composição, bem como métodos de reamostragem;
métodos de aproximação e integração; quadratura Gaussiana, integração de
Monte Carlo e quadratura adaptativa; métodos de Monte Carlo em sentido
amplo; amostradores MCMC, com ênfase em Gibbs e Metropolis--Hastings;
otimização numérica via Newton--Raphson, Fisher scoring e quase-Newton,
além do algoritmo EM; Bootstrap e Jackknife; diagnóstico de
convergência; e aspectos computacionais em problemas práticos, com foco
em implementação eficiente, estabilidade numérica e reprodutibilidade
dos resultados.

Esperamos que este material sirva não apenas como texto-base para as
disciplinas Estatística Computacional (graduação em Estatística) e
Métodos Computacionais em Estatística (Mestrado-PPGMMQ), mas também como
suporte para aqueles que desejam programar com qualidade na área de
Estatística.

\bookmarksetup{startatroot}

\chapter{Introdução}\label{introduuxe7uxe3o}

A simulação tem um papel preponderante na estatística moderna, e suas
vantagens no ensino de Estatística são conhecidas há muito tempo. Em um
de seus primeiros números, o periódico \emph{Teaching Statistics}
publicou artigos que aludem precisamente a isso. Thomas e Moore (1980)
afirmaram que ``a introdução do computador na sala de aula escolar
trouxe uma nova técnica para o ensino, a técnica da simulação''.
Zieffler e Garfield (2007) e Tintle et al. (2015) discutem o papel e a
importância da aprendizagem baseada em simulação no currículo de
graduação em Estatística. No entanto, outros autores (por exemplo,
Hodgson e Burke 2000) discutem alguns problemas que podem surgir ao
ensinar uma disciplina por meio de simulação, a saber, o desenvolvimento
de certos equívocos na mente dos estudantes (Martins 2018).

\bookmarksetup{startatroot}

\chapter{Motivação}\label{motivauxe7uxe3o}

A Estatística, além de lidar com modelos matemáticos rigorosos, também é
permeada por situações em que a intuição humana falha de maneira
sistemática. Um exemplo clássico é o \textbf{problema do aniversário},
que há décadas desperta curiosidade entre estudantes e pesquisadores.

\begin{tcolorbox}[enhanced jigsaw, coltitle=black, bottomtitle=1mm, toprule=.15mm, arc=.35mm, colframe=quarto-callout-tip-color-frame, breakable, opacityback=0, bottomrule=.15mm, rightrule=.15mm, titlerule=0mm, toptitle=1mm, title=\textcolor{quarto-callout-tip-color}{\faLightbulb}\hspace{0.5em}{Enuciado}, leftrule=.75mm, opacitybacktitle=0.6, left=2mm, colback=white, colbacktitle=quarto-callout-tip-color!10!white]

\emph{Em uma sala com \(r\) pessoas, qual a probabilidade de que pelo
menos duas delas compartilhem o mesmo aniversário?}

\end{tcolorbox}

Um resultado surpreendente é: com apenas \textbf{23 pessoas} em uma
sala, a probabilidade de que haja pelo menos uma coincidência de
aniversários já é \textbf{superior a 50\%}. Esse resultado é tão
interessante que pode ser uma porta de entrada natural para discutir a
diferença entre \textbf{probabilidade teórica} e \textbf{evidência
empírica obtida por simulação}.

\section{Da teoria à simulação}\label{da-teoria-uxe0-simulauxe7uxe3o}

Do ponto de vista teórico, a probabilidade de que todos os aniversários
sejam distintos entre \(r\) pessoas é

\[\Pr(\text{todos distintos}) \;=\; \prod_{i=1}^{r-1} \frac{365-i}{365}
\;=\; \left(1 - \frac{1}{365}\right)\!\left(1 - \frac{2}{365}\right)\!\cdots\!\left(1 - \frac{r-1}{365}\right).\]

Logo, a probabilidade de pelo menos uma coincidência é

\[p_r = 1 - \Pr(\text{todos distintos}).\]

Esse produto é conceitualmente claro, mas fica pouco manejável
mentalmente para \(k\) moderados. É aqui que a \textbf{simulação
computacional} pode entrar como aliada didática e científica.

\section{Um atalho analítico útil}\label{um-atalho-analuxedtico-uxfatil}

O produto acima admite uma \textbf{aproximação exponencial simples e
acurada}, obtida tomando logaritmo e usando a expansão para argumentos
pequenos:

\[\ln(1 - x) = -x + o(x), \quad (x \to 0). \] Aplicando ao produto,

\[
\begin{aligned}
\ln\!\big(1 - p_r\big)
\;&=\;
\sum_{i=1}^{r-1} \ln\!\left(1 - \frac{i}{365}\right)\\
\;&\approx\;
-\,\sum_{i=1}^{r-1} \frac{i}{365}
\;=\;
-\,\frac{1 + 2 + \cdots + (r-1)}{365}
\;=\;
-\,\frac{r(r-1)}{2 \cdot 365}.\end{aligned}\]

Exponentiando e isolando \(p_r\), obtemos a aproximação

\[
p_r \;\approx\; 1 - \exp\!\left\{-\,\frac{r(r-1)}{730}\right\}.
\]

Essa fórmula tem três virtudes didáticas:

\begin{enumerate}
\def\labelenumi{\arabic{enumi})}
\item
  \textbf{Clareza}: exibe explicitamente o papel do número de pares
  \(\binom{r}{2}\).
\item
  \textbf{Rapidez}: permite cálculos aproximados para valores de \(r\)
  de interesse.
\item
  \textbf{Boas aproximações} já para \(r\) na casa das dezenas.
\end{enumerate}

\begin{tcolorbox}[enhanced jigsaw, coltitle=black, bottomtitle=1mm, toprule=.15mm, arc=.35mm, colframe=quarto-callout-warning-color-frame, breakable, opacityback=0, bottomrule=.15mm, rightrule=.15mm, titlerule=0mm, toptitle=1mm, title=\textcolor{quarto-callout-warning-color}{\faExclamationTriangle}\hspace{0.5em}{Exemplo Rápido}, leftrule=.75mm, opacitybacktitle=0.6, left=2mm, colback=white, colbacktitle=quarto-callout-warning-color!10!white]

\begin{itemize}
\tightlist
\item
  Para \textbf{23 pessoas}:
\end{itemize}

\[p_{23}^{(\text{aprox})}
  \;=\;
  1 - \exp\!\left\{-\frac{23\cdot22}{730}\right\}
  \;=\;
  1 - \exp\!\{-0.69315\}
  \;\approx\; 0.500,\]

alinhando-se ao resultado clássico de que \textbf{23} pessoas já superam
50\% de chance de coincidência.

\end{tcolorbox}

\section{O papel da simulação}\label{o-papel-da-simulauxe7uxe3o}

A simulação estatística permite reproduzir o experimento de forma
empírica: sorteamos aleatoriamente dias de aniversário para os
indivíduos e verificamos se há repetições. Repetindo o processo milhares
de vezes, obtemos uma estimativa para a probabilidade de coincidência.

Por exemplo, em \textbf{R}:

\begin{Shaded}
\begin{Highlighting}[]
\NormalTok{k }\OtherTok{\textless{}{-}} \DecValTok{23}
\NormalTok{birthdays }\OtherTok{\textless{}{-}} \FunctionTok{sample}\NormalTok{(}\DecValTok{1}\SpecialCharTok{:}\DecValTok{365}\NormalTok{, k, }\AttributeTok{replace =} \ConstantTok{TRUE}\NormalTok{)}
\FunctionTok{any}\NormalTok{(}\FunctionTok{duplicated}\NormalTok{(birthdays))}
\end{Highlighting}
\end{Shaded}

\begin{verbatim}
[1] TRUE
\end{verbatim}

Ao repetir esse procedimento muitas vezes (por exemplo, 10.000
simulações), podemos estimar a proporção de conjuntos com coincidência.
Pela Lei dos Grandes Números, essa estimativa converge para o valor
teórico de aproximadamente 0,507 quando \(k=23\).

\begin{Shaded}
\begin{Highlighting}[]
\FunctionTok{set.seed}\NormalTok{(}\DecValTok{123}\NormalTok{) }\CommentTok{\#reprodutibilidade}

\NormalTok{k }\OtherTok{\textless{}{-}} \DecValTok{23}
\NormalTok{B }\OtherTok{\textless{}{-}} \DecValTok{10000}

\NormalTok{acertos }\OtherTok{\textless{}{-}} \DecValTok{0}\NormalTok{L}
\NormalTok{i }\OtherTok{\textless{}{-}} \DecValTok{0}\NormalTok{L}

\ControlFlowTok{repeat}\NormalTok{ \{}
\NormalTok{  i }\OtherTok{\textless{}{-}}\NormalTok{ i }\SpecialCharTok{+} \DecValTok{1}\NormalTok{L}
\NormalTok{  bdays }\OtherTok{\textless{}{-}} \FunctionTok{sample}\NormalTok{(}\DecValTok{1}\SpecialCharTok{:}\DecValTok{365}\NormalTok{, k, }\AttributeTok{replace =} \ConstantTok{TRUE}\NormalTok{)}
\NormalTok{  acertos }\OtherTok{\textless{}{-}}\NormalTok{ acertos }\SpecialCharTok{+} \FunctionTok{as.integer}\NormalTok{(}\FunctionTok{any}\NormalTok{(}\FunctionTok{duplicated}\NormalTok{(bdays)))}
  \ControlFlowTok{if}\NormalTok{ (i }\SpecialCharTok{\textgreater{}=}\NormalTok{ B) }\ControlFlowTok{break}
\NormalTok{\}}

\NormalTok{p\_hat }\OtherTok{\textless{}{-}}\NormalTok{ acertos }\SpecialCharTok{/}\NormalTok{ B}
\NormalTok{p\_hat}
\end{Highlighting}
\end{Shaded}

\begin{verbatim}
[1] 0.5073
\end{verbatim}

\section{Atividade: Problema do Aniversário (22
jogadores)}\label{atividade-problema-do-aniversuxe1rio-22-jogadores}

Nesta motivação consideramos um exemplo discutido em Martins (2018) que
é o conhecido e amplamente divulgado problema do aniversário (ver, por
exemplo, Falk 2014). Martins (2018) segue o exemplo de Matthews e Stones
(1998), considerando duas equipes de futebol e, portanto, coincidências
de aniversário entre 22 jogadores. Martins (2018) afirma que um
resultado positivo importante dessa atividade é a discussão que surgirá
naturalmente entre os estudantes, com o professor atuando como mediador.
Além disso, os estudantes adoram jogos e a descoberta prática, e a
simulação facilita o engajamento nessas atividades, ao mesmo tempo que
ilustra resultados que podem ser não intuitivos, bem como teoria geral,
como a \textbf{Lei dos Grandes Números}.

Agora iremos considerar o seguinte problema:

\textbf{O problema:} Em uma partida de futebol, qual é a probabilidade
de que pelo menos dois dos 22 jogadores façam aniversário no mesmo dia?

Em um pais chamado de país do futebol, o contexto é proposital: o
futebol é popular e as probabilidades resultantes são contraintuitivas.
Antes de qualquer cálculo, considere as hipóteses: (i) todos os 365 dias
do ano são igualmente prováveis para qualquer aniversário; (ii) as datas
de aniversário dos jogadores são independentes entre si.

\textbf{Objetivos}

\begin{itemize}
\tightlist
\item
  Estimar, via simulação, a probabilidade de coincidência de
  aniversários.\\
\item
  Relacionar frequência relativa, Lei dos Grandes Números e variação
  amostral.\\
\item
  Comparar o resultado exato e aproximado.\\
\end{itemize}

\textbf{Hipóteses}

\begin{itemize}
\tightlist
\item
  365 dias equiprováveis, datas independentes, ignorar bissexto/gêmeos.
\end{itemize}

\textbf{Materiais}

\begin{itemize}
\tightlist
\item
  R (ou Posit Cloud), roteiro com comandos \texttt{sample()},
  \texttt{table()}, \texttt{mean()}.
\end{itemize}

\section{Exercícios}\label{exercuxedcios}

\begin{enumerate}
\def\labelenumi{\arabic{enumi})}
\item
  Determinar o menor número de pessoas que deve estar em uma sala para
  que se possa apostar, com mais de 50\% de chance de ganhar, que entre
  elas existam pelo menos duas com o mesmo aniversário.
\item
  Determinar o menor número de outras pessoas que deve estar em uma sala
  com você para que se possa apostar, com mais de 50\% de chance de
  ganhar, que pelo menos uma delas tenha o mesmo aniversário que o seu.
\end{enumerate}

\bookmarksetup{startatroot}

\chapter{Números Uniformes}\label{nuxfameros-uniformes}

As simulações, de modo geral, requerem uma base inicial formada por
números aleatórios. Diz-se que uma sequência \(R_1, R_2, \ldots\) é
composta por números aleatórios quando cada termo segue a distribuição
uniforme \(U(0,1)\) e \(R_i\) é independente de \(R_j\) para todo
\(i \neq j\). Embora alguns autores utilizem o termo ``números
aleatórios'' para se referir a variáveis amostradas de qualquer
distribuição, aqui ele será usado exclusivamente para variáveis com
distribuição \(U(0,1)\).

\section{\texorpdfstring{Geração de sequências
\(U(0,1)\)}{Geração de sequências U(0,1)}}\label{gerauxe7uxe3o-de-sequuxeancias-u01}

Uma abordagem é utilizar dispositivos físicos aleatorizadores, como
máquinas que sorteiam números de loteria, roletas ou circuitos
eletrônicos que produzem ``ruído aleatório''. Contudo, tais dispositivos
apresentam desvantagens:

\begin{enumerate}
\def\labelenumi{\arabic{enumi}.}
\tightlist
\item
  \textbf{Baixa velocidade} e dificuldade de integração direta com
  computadores.
\item
  \textbf{Necessidade de reprodutibilidade} da sequência. Por exemplo,
  para verificação de código ou comparação de políticas em um modelo de
  simulação, usando a mesma sequência para reduzir a variância da
  diferença entre resultados.
\end{enumerate}

Uma forma simples de obter reprodutibilidade é armazenar a sequência em
um dispositivo de memória (HD, CD-ROM, livro). De fato, a RAND
Corporation publicou \emph{A Million Random Digits with 100 000 Random
Normal Deviates} (1955). Entretanto, acessar armazenamento externo
milhares ou milhões de vezes torna a simulação lenta.

Assim, a abordagem preferida é \textbf{gerar números pseudoaleatórios em
tempo de execução}, via recorrências determinísticas sobre inteiros.
Isso permite:

\begin{itemize}
\tightlist
\item
  Geração rápida;
\item
  Eliminação do problema de armazenamento;
\item
  Reprodutibilidade controlada.
\end{itemize}

Entretanto, a escolha inadequada da recorrência pode gerar sequências
com baixa qualidade estatística.

\section{Geradores Congruenciais
Lineares}\label{geradores-congruenciais-lineares}

Um \textbf{Gerador Congruencial Linear (LGC)} produz uma sequência de
inteiros não negativos \(X_i\), \(i = 1, 2, \ldots\), por meio da
relação de recorrência:

\[X_i = (a X_{i-1} + c) \bmod m, \quad i = 1, 2, \ldots,\] em que
\(a > 0\) é o multiplicador, \(X_0 \ge 0\) é a \emph{semente}
(\emph{seed}), \(c \ge 0\) é o incremento e \(m > 0\) é o módulo.

Os valores \(a, c, X_0\) estão no intervalo \([0, m-1]\). O número
pseudoaleatório \(R_i\) é obtido por:

\[R_i = \frac{X_i}{m}, \quad R_i \in (0,1).\]

Se \(m\) for suficientemente grande, os valores discretos
\(0/m, 1/m, \ldots, (m-1)/m\) são tão próximos que \(R_i\) pode ser
tratado como variável contínua.

\subsection{Exemplo}\label{exemplo}

Seja o gerador:

\[X_i = (9 X_{i-1} + 3) \bmod 24, \quad i \geq 1.\]

Escolhendo \(X_0 = 3\):

\[X_1 = (9 \times 3 + 3) \bmod 24 = 14\]

\[X_2 = (9 \times 14 + 3) \bmod 24 = 1\]

e assim por diante.

A sequência \(R_i = X_i / 24\) gerada terá período \(\ell = 24\).

\subsection{Implementação em R}\label{implementauxe7uxe3o-em-r}

\begin{Shaded}
\begin{Highlighting}[]
\CommentTok{\# Função LCG genérica}
\NormalTok{lcg }\OtherTok{\textless{}{-}} \ControlFlowTok{function}\NormalTok{(a, c, m, seed, n) \{}
\NormalTok{  x }\OtherTok{\textless{}{-}} \FunctionTok{numeric}\NormalTok{(n)}
\NormalTok{  x[}\DecValTok{1}\NormalTok{] }\OtherTok{\textless{}{-}}\NormalTok{ seed}
  \ControlFlowTok{for}\NormalTok{ (i }\ControlFlowTok{in} \DecValTok{2}\SpecialCharTok{:}\NormalTok{n) \{}
\NormalTok{    x[i] }\OtherTok{\textless{}{-}}\NormalTok{ (a }\SpecialCharTok{*}\NormalTok{ x[i}\DecValTok{{-}1}\NormalTok{] }\SpecialCharTok{+}\NormalTok{ c) }\SpecialCharTok{\%\%}\NormalTok{ m}
\NormalTok{  \}}
\NormalTok{  r }\OtherTok{\textless{}{-}}\NormalTok{ x }\SpecialCharTok{/}\NormalTok{ m}
  \FunctionTok{return}\NormalTok{(}\FunctionTok{list}\NormalTok{(}\AttributeTok{X =}\NormalTok{ x, }\AttributeTok{R =}\NormalTok{ r))}
\NormalTok{\}}

\CommentTok{\# Exemplo com a = 9, c = 3, m = 24, seed = 3}
\NormalTok{resultado }\OtherTok{\textless{}{-}} \FunctionTok{lcg}\NormalTok{(}\AttributeTok{a =} \DecValTok{9}\NormalTok{, }\AttributeTok{c =} \DecValTok{3}\NormalTok{, }\AttributeTok{m =} \DecValTok{24}\NormalTok{, }\AttributeTok{seed =} \DecValTok{3}\NormalTok{, }\AttributeTok{n =} \DecValTok{20}\NormalTok{)}
\NormalTok{resultado}\SpecialCharTok{$}\NormalTok{X}
\end{Highlighting}
\end{Shaded}

\begin{verbatim}
 [1]  3  6  9 12 15 18 21  0  3  6  9 12 15 18 21  0  3  6  9 12
\end{verbatim}

\begin{Shaded}
\begin{Highlighting}[]
\NormalTok{resultado}\SpecialCharTok{$}\NormalTok{R}
\end{Highlighting}
\end{Shaded}

\begin{verbatim}
 [1] 0.125 0.250 0.375 0.500 0.625 0.750 0.875 0.000 0.125 0.250 0.375 0.500
[13] 0.625 0.750 0.875 0.000 0.125 0.250 0.375 0.500
\end{verbatim}

\section{Geradores Congruenciais Lineares
Mistos}\label{geradores-congruenciais-lineares-mistos}

Nos LCGs \textbf{mistos} temos \(c>0\). Uma escolha prática é
\(m = 2^b\), onde \(b\) é o número de bits utilizável para inteiros
positivos na arquitetura/linguagem. Em muitos ambientes, inteiros usam
32 bits (um para o sinal), implicando \(b=31\) e intervalo
\([-2^{31}, 2^{31}-1]\).

Quando \(m=2^b\), obtemos \textbf{período completo} (\(\ell=m\)) se:

\begin{enumerate}
\def\labelenumi{\arabic{enumi})}
\tightlist
\item
  \(c\) é \textbf{ímpar} (garante \(\gcd(c,m)=1\));\\
\item
  \(a-1\) é múltiplo de todos os fatores primos de \(m\) e também de 4
  (como \(m\) é potência de 2).
\end{enumerate}

Essa é a razão de geradores simples com \(m=2^b\), \(c\) ímpar e
\(a \equiv 1 \pmod 4\) atingirem \(\ell=m\).

\subsection{Questão de estouro e aritmética
modular}\label{questuxe3o-de-estouro-e-aritmuxe9tica-modular}

Em linguagens com inteiros limitados, calcular \(aX_{i-1}+c\) pode
\textbf{transbordar}. Soluções comuns:

\begin{itemize}
\tightlist
\item
  usar precisão estendida (64 bits) ou bibliotecas de inteiros grandes;
\item
  empregar \textbf{truques de aritmética modular} (como o método de
  Schrage) para evitar overflow;
\item
  trabalhar com módulo \(m=2^b\) e aproveitar o ``wrap'' de bits.
\end{itemize}

A seguir, implementamos LCG misto com \(m=2^{31}\), \(a=906185749\),
\(c=1\). Parâmetros com boas propriedades estatísticas relatadas na
literatura.

\subsection{Implementação em R (com segurança de
overflow)}\label{implementauxe7uxe3o-em-r-com-seguranuxe7a-de-overflow}

Para garantir a correção do módulo com inteiros grandes, usaremos
\texttt{bit64} (inteiros de 64 bits) e normalizaremos para \((0,1)\).

\begin{Shaded}
\begin{Highlighting}[]
\CommentTok{\#if (!requireNamespace("bit64", quietly = TRUE)) \{}
\CommentTok{\#  install.packages("bit64")}
\CommentTok{\#\}}

\FunctionTok{library}\NormalTok{(bit64)}

\NormalTok{lcg\_misto }\OtherTok{\textless{}{-}} \ControlFlowTok{function}\NormalTok{(n, }\AttributeTok{seed =} \DecValTok{3456}\NormalTok{L,}
                      \AttributeTok{a =} \DecValTok{906185749}\NormalTok{L,}
                      \AttributeTok{c =} \DecValTok{1}\NormalTok{L,}
                      \AttributeTok{m =}\NormalTok{ bit64}\SpecialCharTok{::}\FunctionTok{as.integer64}\NormalTok{(}\DecValTok{2}\NormalTok{)}\SpecialCharTok{\^{}}\DecValTok{31}\NormalTok{) \{}
  \CommentTok{\# Trabalha em integer64 para evitar perda de precisão}
\NormalTok{  x }\OtherTok{\textless{}{-}}\NormalTok{ bit64}\SpecialCharTok{::}\FunctionTok{as.integer64}\NormalTok{(seed)}
\NormalTok{  outX }\OtherTok{\textless{}{-}}\NormalTok{ bit64}\SpecialCharTok{::}\FunctionTok{integer64}\NormalTok{(n)}
\NormalTok{  outR }\OtherTok{\textless{}{-}} \FunctionTok{numeric}\NormalTok{(n)}
\NormalTok{  outX[}\DecValTok{1}\NormalTok{] }\OtherTok{\textless{}{-}}\NormalTok{ x}
\NormalTok{  outR[}\DecValTok{1}\NormalTok{] }\OtherTok{\textless{}{-}} \FunctionTok{as.double}\NormalTok{(x) }\SpecialCharTok{/} \FunctionTok{as.double}\NormalTok{(m)}
  \ControlFlowTok{for}\NormalTok{ (i }\ControlFlowTok{in} \DecValTok{2}\SpecialCharTok{:}\NormalTok{n) \{}
\NormalTok{    x }\OtherTok{\textless{}{-}}\NormalTok{ (bit64}\SpecialCharTok{::}\FunctionTok{as.integer64}\NormalTok{(a) }\SpecialCharTok{*}\NormalTok{ x }\SpecialCharTok{+}\NormalTok{ bit64}\SpecialCharTok{::}\FunctionTok{as.integer64}\NormalTok{(c)) }\SpecialCharTok{\%\%}\NormalTok{ m}
\NormalTok{    outX[i] }\OtherTok{\textless{}{-}}\NormalTok{ x}
\NormalTok{    outR[i] }\OtherTok{\textless{}{-}} \FunctionTok{as.double}\NormalTok{(x) }\SpecialCharTok{/} \FunctionTok{as.double}\NormalTok{(m)}
\NormalTok{  \}}
  \FunctionTok{list}\NormalTok{(}\AttributeTok{X =}\NormalTok{ outX, }\AttributeTok{R =}\NormalTok{ outR)}
\NormalTok{\}}

\CommentTok{\# Exemplo: primeiros 5 números com seed = 3456}
\FunctionTok{set.seed}\NormalTok{(}\ConstantTok{NULL}\NormalTok{)}
\NormalTok{g1 }\OtherTok{\textless{}{-}} \FunctionTok{lcg\_misto}\NormalTok{(}\AttributeTok{n =} \DecValTok{5}\NormalTok{, }\AttributeTok{seed =} \DecValTok{3456}\NormalTok{L)}
\NormalTok{g1}\SpecialCharTok{$}\NormalTok{X}
\end{Highlighting}
\end{Shaded}

\begin{verbatim}
integer64
[1] 3456       746789761  460230038  1591485775 1024426876
\end{verbatim}

\begin{Shaded}
\begin{Highlighting}[]
\NormalTok{g1}\SpecialCharTok{$}\NormalTok{R}
\end{Highlighting}
\end{Shaded}

\begin{verbatim}
[1] 1.609325e-06 3.477511e-01 2.143113e-01 7.410933e-01 4.770359e-01
\end{verbatim}

\section{Geradores Congruenciais Lineares
Multiplicativos}\label{geradores-congruenciais-lineares-multiplicativos}

No caso \textbf{multiplicativo}, temos \(c = 0\), e a recorrência fica:

\[
X_i = (a X_{i-1}) \bmod m
\]

\subsection{Características e
restrições}\label{caracteruxedsticas-e-restriuxe7uxf5es}

\begin{itemize}
\tightlist
\item
  Se \(X_i = 0\) em algum passo, toda a sequência futura será zero ---
  portanto \(X_0 \neq 0\).
\item
  Se \(a = 1\), a sequência é constante --- também deve ser evitado.
\item
  O \textbf{período máximo} possível é \(m - 1\), e ele só é atingido
  quando:

  \begin{enumerate}
  \def\labelenumi{\arabic{enumi}.}
  \tightlist
  \item
    \(m\) é primo;
  \item
    \(a\) é uma \textbf{raiz primitiva} módulo \(m\).
  \end{enumerate}
\end{itemize}

\subsection{Definição de raiz
primitiva}\label{definiuxe7uxe3o-de-raiz-primitiva}

Um número \(a\) é raiz primitiva módulo \(m\) se seus poderes geram
todos os inteiros não nulos módulo \(m\).\\
Matematicamente, \(a\) satisfaz:

\[
m \nmid a^{(m-1)/q} - 1, \quad \forall q \text{ primo que divide } m-1
\]

Esse tipo de gerador é chamado \textbf{Gerador de Módulo Primo e Período
Máximo}.

\begin{center}\rule{0.5\linewidth}{0.5pt}\end{center}

\subsection{Exemplo de implementação em
R}\label{exemplo-de-implementauxe7uxe3o-em-r}

A seguir, implementamos um gerador multiplicativo com módulo primo
\(m = 2^{31} - 1\) (primo de Mersenne) e multiplicador
\(a = 630360016\), conhecido por apresentar boas propriedades
estatísticas.

\begin{Shaded}
\begin{Highlighting}[]
\ControlFlowTok{if}\NormalTok{ (}\SpecialCharTok{!}\FunctionTok{requireNamespace}\NormalTok{(}\StringTok{"gmp"}\NormalTok{, }\AttributeTok{quietly =} \ConstantTok{TRUE}\NormalTok{)) \{}
  \FunctionTok{install.packages}\NormalTok{(}\StringTok{"gmp"}\NormalTok{)}
\NormalTok{\}}
\FunctionTok{library}\NormalTok{(gmp)}

\NormalTok{lcg\_mult\_primo }\OtherTok{\textless{}{-}} \ControlFlowTok{function}\NormalTok{(n, seed, }\AttributeTok{a =} \DecValTok{630360016}\NormalTok{, }\AttributeTok{m =} \DecValTok{2147483647}\NormalTok{) \{}
\NormalTok{  A }\OtherTok{\textless{}{-}} \FunctionTok{as.bigz}\NormalTok{(a); M }\OtherTok{\textless{}{-}} \FunctionTok{as.bigz}\NormalTok{(m)}
\NormalTok{  x }\OtherTok{\textless{}{-}} \FunctionTok{as.bigz}\NormalTok{(seed)}
\NormalTok{  X }\OtherTok{\textless{}{-}} \FunctionTok{integer}\NormalTok{(n); R }\OtherTok{\textless{}{-}} \FunctionTok{numeric}\NormalTok{(n)}
  \ControlFlowTok{for}\NormalTok{ (i }\ControlFlowTok{in} \FunctionTok{seq\_len}\NormalTok{(n)) \{}
\NormalTok{    X[i] }\OtherTok{\textless{}{-}} \FunctionTok{as.integer}\NormalTok{(x)    }
\NormalTok{    R[i] }\OtherTok{\textless{}{-}} \FunctionTok{as.numeric}\NormalTok{(x) }\SpecialCharTok{/}\NormalTok{ m    }
\NormalTok{    x }\OtherTok{\textless{}{-}}\NormalTok{ (A }\SpecialCharTok{*}\NormalTok{ x) }\SpecialCharTok{\%\%}\NormalTok{ M            }
\NormalTok{  \}}
  \FunctionTok{list}\NormalTok{(}\AttributeTok{X =}\NormalTok{ X, }\AttributeTok{R =}\NormalTok{ R)}
\NormalTok{\}}

\CommentTok{\# Exemplo: gerar 10 valores}
\NormalTok{g2 }\OtherTok{\textless{}{-}} \FunctionTok{lcg\_mult\_primo}\NormalTok{(}\AttributeTok{n =} \DecValTok{10}\NormalTok{, }\AttributeTok{seed =} \DecValTok{12345}\NormalTok{L)}
\NormalTok{g2}\SpecialCharTok{$}\NormalTok{X}
\end{Highlighting}
\end{Shaded}

\begin{verbatim}
 [1]      12345 1461144439 1646755962  423395703 2041926374  720397004
 [7]  140279311  597861375  629442282  759842328
\end{verbatim}

\begin{Shaded}
\begin{Highlighting}[]
\NormalTok{g2}\SpecialCharTok{$}\NormalTok{R}
\end{Highlighting}
\end{Shaded}

\begin{verbatim}
 [1] 5.748589e-06 6.803984e-01 7.668305e-01 1.971590e-01 9.508461e-01
 [6] 3.354610e-01 6.532264e-02 2.784009e-01 2.931069e-01 3.538292e-01
\end{verbatim}

\section{Uso de Números Aleatórios na Avaliação de
Integrais}\label{uso-de-nuxfameros-aleatuxf3rios-na-avaliauxe7uxe3o-de-integrais}

Uma das primeiras aplicações do uso de números aleatórios foi na
resolução de integrais. Considere uma função \(g(x)\) e suponha que
desejamos calcular uma integral de interesse.

\[\theta = \int\limits_{0}^{1}g(x) dx.\]

Para calcular o valor da integral, observe que, se \(U\) é uma variável
aleatória com distribuição uniforme no intervalo \((0, 1)\), , então
podemos reescrever a integral da seguinte forma:

\[\theta = E[g(U)].\] Se \(U_1, U_2, \dots,  U_n\) são variáveis
aleatórias independentes e uniformes em \((0, 1)\), então as variáveis
\(g(U_1), g(U_2), \dots,  g(U_n)\) são indepedentes e identicamente
distribuídas, todas com esperança igual \(\theta\) (o valor da
integral). Assim, pelo \textbf{Teorema da Lei Forte dos Grandes
Números}, temos que, com probabilidade 1,

\[\frac{1}{n}\sum\limits_{i=1}^{n}g(U_i) \to \theta \quad \text{quando} \quad n \to \infty.\]

Assim, podemos aproximar o valor da integral gerando um grande número de
pontos aleatórios \(u_i\) no intervalo \((0, 1)\) e tomando como
estimativa a média dos valores \(g(u_i)\). Esse procedimento de
aproximação de integrais é conhecido como método de \textbf{Monte
Carlo}.

Se quisermos calcular uma integral mais geral, podemos aplicar a mesma
ideia: transformar o problema em uma esperança matemática e, em seguida,
aproximá-la por meio de médias amostrais obtidas a partir de números
aleatórios. Considere:

\[\theta = \int\limits_{a}^{b}g(x) dx.\]

Se quisermos calcular a integral em um intervalo genérico \((a, b)\),
fazemos a substituição

\[u = \frac{x - a}{b-a}, \quad du = \frac{dx}{b-a},\]

o que nos permite reescrevê-la como

\[\theta = \int\limits_{a}^{b}g(x) dx = (b-a)\int\limits_{0}^{1}g(a + (b-a)u) du.\]

Definindo

\[h(u) = (b-a)g(a + (b-a)u),\]

temos

\[\theta = \int\limits_{0}^{1}h(u) du.\]

Assim, podemos aproximar \(\theta\) gerando números aleatórios
\(u_1, u_2, \dots, u_n \sim U(0, 1)\) e tomando como estimativa a média

\[\theta \approx \frac{1}{n}\sum\limits_{i=1}^{n}h(u_i).\] Agora, se
nosso objetivo é calcular a integral:

\[\theta = \int\limits_{0}^{\infty}g(x) dx.\]

Fazendo a mudança de variável

\[u = \frac{1}{x+1}, \quad du = \frac{-dx}{(x+1)^2} = -u^2 dx.\]

Logo,

\[dx = -\frac{du}{u^2},\]

e a integral resultante é

\[\theta = \int\limits_{0}^{1}h(u) du,\]

com

\[h(u) = \frac{g\left(\frac{1}{u} - 1\right)}{u^2}.\]

A utilidade de empregar números aleatórios para aproximar integrais
torna-se ainda mais evidente no caso de integrais multidimensionais.
Suponha que \(g\) seja uma função com argumento \(n\)-dimensional e que
estejamos interessados em calcular:

\[\theta = \int\limits_{0}^{1}\int\limits_{0}^{1}\dots\int\limits_{0}^{1} g(x_1, x_2, \dots, x_n)dx_1 dx_2 \dots dx_n.\]

Observe que \(\theta\) pode ser expresso como o seguinte valor esperado:

\[\theta = E[g(U_1, U_2, \dots, U_n)],\]

em que \(U_1, U_2, \dots, U_n\) são variáveis aleatória independentes
uniformente distribuídas no intervalo \((0, 1)\). Assim, se gerarmos
\(k\) conjuntos independentes, cada um formado por \(n\) variáveis
aleatórias independentes com distribuição uniforme em \((0, 1)\), então
as variáveis

\[g(U_{i1}, U_{i2}, \dots, U_{in}), \quad i = 1, 2, \dots, k,\] serão
independentes e identicamente distribuídas, todas com esperança igual a
\(\theta\) (o valor da integral). Portanto, podemos aproximar \(\theta\)
por meio da média amostral:

\[\theta \approx \frac{1}{k}\sum\limits_{i=1}^{k}g(U_{i1}, U_{i2}, \dots, U_{in}).\]

\bookmarksetup{startatroot}

\chapter{Números Pseudoaleatórios}\label{nuxfameros-pseudoaleatuxf3rios}

\section{Introdução}\label{introduuxe7uxe3o-1}

\section{Método da Transformação
Inversa}\label{muxe9todo-da-transformauxe7uxe3o-inversa}

Suponha que desejamos gerar o valor de uma variável aleatória discreta
\(X\) com função de massa de probabilidade (f.m.p):

\[\Pr(X = x_j) = p_j, \quad j = 0, 1, \dots, \quad \sum\limits_j p_j = 1.\]

\begin{tcolorbox}[enhanced jigsaw, toprule=.15mm, arc=.35mm, colframe=quarto-callout-important-color-frame, breakable, opacityback=0, rightrule=.15mm, bottomrule=.15mm, left=2mm, leftrule=.75mm, colback=white]

\vspace{-3mm}\textbf{\textbf{Interesse:} Gerar valores da variável aleatória \(X\) com
distribuição:}\vspace{3mm}

\begin{longtable}[]{@{}ll@{}}
\toprule\noalign{}
\(X\) & \(P(X=x_j)\) \\
\midrule\noalign{}
\endhead
\bottomrule\noalign{}
\endlastfoot
\(x_0\) & \(p_0\) \\
\(x_1\) & \(p_1\) \\
\(x_2\) & \(p_2\) \\
\(\vdots\) & \(\vdots\) \\
\(x_j\) & \(p_j\) \\
\(\vdots\) & \(\vdots\) \\
\end{longtable}

\end{tcolorbox}

Para realizar isso, gere um valor \(u\) de \(U\sim(0,1)\) e atribua o
valor \(x_j\), \(j=0,1,\ldots\), a \(X\) conforme as condições abaixo:

\[X = \begin{cases}
x_0, & \text{se } u < p_0, \\
x_1, & \text{se } p_0 \leq u < p_0+p_1, \\
x_2, & \text{se } p_0+p_1 \leq u < p_0+p_1+p_2, \\
x_3, & \text{se } p_0+p_1+p_2 \leq u < p_0+p_1+p_2+p_3, \\
\vdots & \\
x_j, & \text{se } \displaystyle \sum_{i=0}^{j-1} p_i \leq u < \sum_{i=0}^{j} p_i, \\
\vdots &
\end{cases}\]

Como, para \(0 < a < b < 1, \Pr(a \leq U < b) = b-a\), temos que:

\[\Pr(X = x_j) = \Pr\left(\sum\limits_{i=0}^{j-1}p_i \leq U < \sum\limits_{i=0}^{j}p_i\right) = p_j,\]
e, portanto, \(X\) possui a distribuição desejada.

\begin{tcolorbox}[enhanced jigsaw, toprule=.15mm, arc=.35mm, colframe=quarto-callout-important-color-frame, breakable, opacityback=0, rightrule=.15mm, bottomrule=.15mm, left=2mm, leftrule=.75mm, colback=white]

\vspace{-3mm}\textbf{Observações:}\vspace{3mm}

\begin{enumerate}
\def\labelenumi{\arabic{enumi}.}
\tightlist
\item
  Algoritmo - O procedimento anterior pode ser escrito como:
\end{enumerate}

Passo 1. Gerar um valor \(u \sim U(0,1)\).

Passo 2. Se \(u < p_0\), então \(X = x_0\); caso contrário:

\begin{itemize}
\tightlist
\item
  se \(u < p_0 + p_1\), então \(X = x_1\); caso contrário:\\
\item
  se \(u < p_0 + p_1 + p_2\), então \(X = x_2\); caso contrário:\\
  \(\vdots\)
\end{itemize}

Passo 3. Repetir os passos 1 e 2 \(n\) vezes, onde \(n\) é o tamanho da
amostra.

\begin{enumerate}
\def\labelenumi{\arabic{enumi}.}
\setcounter{enumi}{1}
\tightlist
\item
  Se os valores, \(x_i, i \geq 0\), são ordenados de modo que
  \(x_0 < x_1 < \dots,\) e de denotamos \(F\) como função de
  distribuição de \(X\), então \(F(x_j) = \sum\limits_{i=1}^{j}p_i\).
  Assim,
\end{enumerate}

\[X\; \text{será igual a}\; x_j\; \text{se}\; F(x_{j-1}) < U < F(x_j).\]

\end{tcolorbox}

Em outras palavras, após gerar um número aleatório \(U\), determinamos o
valor de \(X\) encontrando o intervalo \([F(x_{j-1}), F(x_j))\) no qual
\(U\) se encontra (ou, de forma equivalente, encontrando o inverso de
\(F(U)\)). É por essa razão que o procedimento acima é denominado
\emph{método da transformada inversa discreta} para gerar \(X\).

\begin{tcolorbox}[enhanced jigsaw, coltitle=black, bottomtitle=1mm, toprule=.15mm, arc=.35mm, colframe=quarto-callout-note-color-frame, breakable, opacityback=0, bottomrule=.15mm, rightrule=.15mm, titlerule=0mm, toptitle=1mm, title=\textcolor{quarto-callout-note-color}{\faInfo}\hspace{0.5em}{Nota}, leftrule=.75mm, opacitybacktitle=0.6, left=2mm, colback=white, colbacktitle=quarto-callout-note-color!10!white]

O tempo necessário para gerar uma variável aleatória discreta por esse
método é proporcional ao número de intervalos que devem ser pesquisados.
Por essa razão, às vezes é vantajoso considerar os possíveis valores
\(x_j\) de \(X\) em ordem decrescente das probabilidades \(p_j\).

\end{tcolorbox}

\subsection*{Exemplo 1}\label{exemplo-1}
\addcontentsline{toc}{subsection}{Exemplo 1}

Simular \(n\) valores de \(X\) tal que \(p_1 = 0.20\), \(p_2 = 0.15\),
\(p_3 = 0.25\), \(p_4 = 0.40\), em que \(p_j = P(X=j)\).

\begin{tcolorbox}[enhanced jigsaw, toprule=.15mm, arc=.35mm, colframe=quarto-callout-warning-color-frame, breakable, opacityback=0, rightrule=.15mm, bottomrule=.15mm, left=2mm, leftrule=.75mm, colback=white]

\vspace{-3mm}\textbf{Algoritmo 1}\vspace{3mm}

Passo 1. Gerar \(n\) valores \(u \sim U(0,1)\).

Passo 2. Para cada \(u\):

\begin{itemize}
\item
  Se \(u < 0.20\), então \(X = 1\).
\item
  Caso contrário, se \(u < 0.35\), então \(X = 2\).
\item
  Caso contrário, se \(u < 0.60\), então \(X = 3\).
\item
  Caso contrário (\(u \geq 0.60\)), então \(X = 4\).
\end{itemize}

Passo 3. Repetir \(n\) vezes os passos 1 e 2.

\end{tcolorbox}

A seguir, são apresentados dos códigos escritos na linguagem \texttt{R}:

\begin{Shaded}
\begin{Highlighting}[]
\CommentTok{\# Proposta 1}
\NormalTok{n }\OtherTok{\textless{}{-}} \DecValTok{10000}
\NormalTok{x }\OtherTok{\textless{}{-}} \DecValTok{1}\SpecialCharTok{:}\DecValTok{4} 
\NormalTok{p }\OtherTok{\textless{}{-}} \FunctionTok{c}\NormalTok{(}\FloatTok{0.20}\NormalTok{, }\FloatTok{0.15}\NormalTok{, }\FloatTok{0.25}\NormalTok{, }\FloatTok{0.40}\NormalTok{)}
\NormalTok{pa }\OtherTok{\textless{}{-}} \FunctionTok{cumsum}\NormalTok{(p) }\CommentTok{\# probabilidades acumuladas}
\NormalTok{a  }\OtherTok{\textless{}{-}} \FunctionTok{c}\NormalTok{() }\CommentTok{\# vetor para a amostra gerada de X}

\ControlFlowTok{for}\NormalTok{(i }\ControlFlowTok{in} \DecValTok{1}\SpecialCharTok{:}\NormalTok{n)\{}
\NormalTok{  u }\OtherTok{\textless{}{-}} \FunctionTok{runif}\NormalTok{(}\DecValTok{1}\NormalTok{)}
  \ControlFlowTok{if}\NormalTok{ (u }\SpecialCharTok{\textless{}}\NormalTok{ pa[}\DecValTok{1}\NormalTok{]) \{}
\NormalTok{    a[i] }\OtherTok{\textless{}{-}}\NormalTok{ x[}\DecValTok{1}\NormalTok{]}
\NormalTok{  \} }\ControlFlowTok{else}\NormalTok{ \{ }
      \ControlFlowTok{if}\NormalTok{ (u }\SpecialCharTok{\textless{}}\NormalTok{ pa[}\DecValTok{2}\NormalTok{]) \{}
\NormalTok{        a[i] }\OtherTok{\textless{}{-}}\NormalTok{ x[}\DecValTok{2}\NormalTok{]}
\NormalTok{      \} }\ControlFlowTok{else}\NormalTok{ \{ }\ControlFlowTok{if}\NormalTok{ (u }\SpecialCharTok{\textless{}}\NormalTok{ pa[}\DecValTok{3}\NormalTok{]) \{}
\NormalTok{        a[i] }\OtherTok{\textless{}{-}}\NormalTok{ x[}\DecValTok{3}\NormalTok{]}
\NormalTok{      \} }\ControlFlowTok{else}\NormalTok{ \{ }
\NormalTok{          a[i] }\OtherTok{\textless{}{-}}\NormalTok{ x[}\DecValTok{4}\NormalTok{]}
\NormalTok{          \}}
\NormalTok{      \}}
\NormalTok{    \}}
\NormalTok{\}}
\FunctionTok{table}\NormalTok{(a)}\SpecialCharTok{/}\NormalTok{n }\CommentTok{\# tabela de proporções}
\end{Highlighting}
\end{Shaded}

\begin{verbatim}
a
     1      2      3      4 
0.1992 0.1533 0.2465 0.4010 
\end{verbatim}

\begin{Shaded}
\begin{Highlighting}[]
\CommentTok{\# Proposta 2}
\NormalTok{n }\OtherTok{\textless{}{-}} \DecValTok{10000}\NormalTok{; }
\NormalTok{x }\OtherTok{\textless{}{-}} \DecValTok{1}\SpecialCharTok{:}\DecValTok{4}\NormalTok{; }
\NormalTok{p }\OtherTok{\textless{}{-}} \FunctionTok{c}\NormalTok{(}\FloatTok{0.20}\NormalTok{, }\FloatTok{0.15}\NormalTok{, }\FloatTok{0.25}\NormalTok{, }\FloatTok{0.40}\NormalTok{)}
\NormalTok{pa }\OtherTok{\textless{}{-}} \FunctionTok{cumsum}\NormalTok{(p) }\CommentTok{\# probabilidades acumuladas}
\NormalTok{a  }\OtherTok{\textless{}{-}} \FunctionTok{c}\NormalTok{() }\CommentTok{\# vetor para a amostra gerada de X}
        
\ControlFlowTok{for}\NormalTok{(i }\ControlFlowTok{in} \DecValTok{1}\SpecialCharTok{:}\NormalTok{n)\{}
\NormalTok{  u }\OtherTok{=} \FunctionTok{runif}\NormalTok{(}\DecValTok{1}\NormalTok{)}
  \FunctionTok{ifelse}\NormalTok{(u }\SpecialCharTok{\textless{}}\NormalTok{ pa[}\DecValTok{1}\NormalTok{], a[i] }\OtherTok{\textless{}{-}}\NormalTok{ x[}\DecValTok{1}\NormalTok{], }
    \FunctionTok{ifelse}\NormalTok{(u }\SpecialCharTok{\textless{}}\NormalTok{ pa[}\DecValTok{2}\NormalTok{], a[i] }\OtherTok{\textless{}{-}}\NormalTok{ x[}\DecValTok{2}\NormalTok{],}
      \FunctionTok{ifelse}\NormalTok{(u }\SpecialCharTok{\textless{}}\NormalTok{ pa[}\DecValTok{3}\NormalTok{], a[i] }\OtherTok{\textless{}{-}}\NormalTok{ x[}\DecValTok{3}\NormalTok{], a[i] }\OtherTok{\textless{}{-}}\NormalTok{ x[}\DecValTok{4}\NormalTok{])))}
\NormalTok{\}}
\FunctionTok{table}\NormalTok{(a)}\SpecialCharTok{/}\NormalTok{n }\CommentTok{\# tabela de proporções}
\end{Highlighting}
\end{Shaded}

\begin{verbatim}
a
     1      2      3      4 
0.1937 0.1544 0.2537 0.3982 
\end{verbatim}

\begin{tcolorbox}[enhanced jigsaw, toprule=.15mm, arc=.35mm, colframe=quarto-callout-warning-color-frame, breakable, opacityback=0, rightrule=.15mm, bottomrule=.15mm, left=2mm, leftrule=.75mm, colback=white]

\vspace{-3mm}\textbf{Algoritmo 2}\vspace{3mm}

Passo 1. Gerar um valor \(u \sim U(0,1)\).

Passo 2. Para cada \(u\):

\begin{itemize}
\item
  Se \(u < 0.40\), então \(X = 4\);
\item
  Caso contrário, se \(u < 0.65\), então \(X = 3\);
\item
  Caso contrário, se \(u < 0.85\), então \(X = 1\);
\item
  Caso contrário \(u \geq 0.85\), então \(X = 2\).
\end{itemize}

Passo 3. Repetir \(n\) vezes os passos 1 e 2.

\end{tcolorbox}

\begin{tcolorbox}[enhanced jigsaw, coltitle=black, bottomtitle=1mm, toprule=.15mm, arc=.35mm, colframe=quarto-callout-important-color-frame, breakable, opacityback=0, bottomrule=.15mm, rightrule=.15mm, titlerule=0mm, toptitle=1mm, title={🔎 \textbf{Observação}:}, leftrule=.75mm, opacitybacktitle=0.6, left=2mm, colback=white, colbacktitle=quarto-callout-important-color!10!white]

A proposta 2 usa a \textbf{ordem decrescente} dos \(p_j\) para otimizar
a busca.

\end{tcolorbox}

\begin{tcolorbox}[enhanced jigsaw, toprule=.15mm, arc=.35mm, colframe=quarto-callout-warning-color-frame, breakable, opacityback=0, rightrule=.15mm, bottomrule=.15mm, left=2mm, leftrule=.75mm, colback=white]

\vspace{-3mm}\textbf{Algoritmo 3 (mudando as desigualdades)}\vspace{3mm}

Passo 1. Gerar \(u \sim U(0,1)\).

Passo 2. Se \(u \leq 0.40\), então \(X = 4\);

\begin{itemize}
\item
  senão, se \(u \leq 0.65\), então \(X = 3\);
\item
  senão, se \(u \leq 0.85\), então \(X = 1\);
\item
  senão (\(u > 0.85\)), então \(X = 2\).
\end{itemize}

Passo 3. Repetir \(n\) vezes os passos 1--2.

\end{tcolorbox}

\begin{tcolorbox}[enhanced jigsaw, coltitle=black, bottomtitle=1mm, toprule=.15mm, arc=.35mm, colframe=quarto-callout-important-color-frame, breakable, opacityback=0, bottomrule=.15mm, rightrule=.15mm, titlerule=0mm, toptitle=1mm, title={🔎 \textbf{Observação}:}, leftrule=.75mm, opacitybacktitle=0.6, left=2mm, colback=white, colbacktitle=quarto-callout-important-color!10!white]

Com \(u \sim U(0,1)\) é contínua, usar \(<\) ou \(\leq\) é equivalente
em termos de distribuição (a probabilidade de \(u\) cair exatamente no
ponto de corte é zero). Em implementação numérica, essa escolha apenas
define para onde vão raríssimos empates nos pontos cortes.

\end{tcolorbox}

\subsection*{Inversa Generalizada}\label{inversa-generalizada}
\addcontentsline{toc}{subsection}{Inversa Generalizada}

Seja \(F\) uma função de distribuição qualquer. A inversa generalizada,
denotada por \(F^{-1}\), é definida da seguinte forma:
\[ F^{-1}(u) = \mbox{inf}\{x \in \mathbb{R}: F(x)\geq u \}\]

\begin{itemize}
\tightlist
\item
  Tanto a inversa de \(F\) como a inversa generalizada estão sendo
  denotadas por \(F^{-1}\).
\item
  Se a inversa de \(F\) existe no sentido usual, ela coincide com a
  inversa generalizada.\\
  Note que, se \(F\) for estritamente crescente e contínua, então
  \(\forall x \in \mathbb{R}\) existirá apenas um \(u\) tal que
  \(F(x) = u\).
\item
  A proposição seguinte dá base para as simulações de v.a. discretas.
\end{itemize}

Seja \(F\) uma função de distribuição qualquer e \(F^{-1}\) sua inversa
generalizada. Temos:
\[F^{-1}(u) \leq x \mbox{ se, e somente se, } u \leq F(x)\]

Se você deseja simular valores de
\Xn  \textbf{com distribuição uniforme discreta}:

\[P(X=j)=1/k, \, \,\, j=1,2,3,\ldots, k.\]

Basta utilizar a Proposições 1 ou a Proposição 2.

\begin{tcolorbox}[enhanced jigsaw, coltitle=black, bottomtitle=1mm, toprule=.15mm, arc=.35mm, colframe=quarto-callout-note-color-frame, breakable, opacityback=0, bottomrule=.15mm, rightrule=.15mm, titlerule=0mm, toptitle=1mm, title={Proposição 1:}, leftrule=.75mm, opacitybacktitle=0.6, left=2mm, colback=white, colbacktitle=quarto-callout-note-color!10!white]

Gere \(U\sim U(0,1)\). Defina \(X=j\) se

\[\frac{j-1}{k}\leq U < \frac{j}{k},\] ou, equivalentemente,

\[(j-1)\leq kU < j.\]

\end{tcolorbox}

\begin{tcolorbox}[enhanced jigsaw, coltitle=black, bottomtitle=1mm, toprule=.15mm, arc=.35mm, colframe=quarto-callout-note-color-frame, breakable, opacityback=0, bottomrule=.15mm, rightrule=.15mm, titlerule=0mm, toptitle=1mm, title={Proposição 2:}, leftrule=.75mm, opacitybacktitle=0.6, left=2mm, colback=white, colbacktitle=quarto-callout-note-color!10!white]

Gere \(U\sim U(0,1)\). Defina \(X=\mbox{Int}(kU)+1\), em que
\(\textrm{Int}(x)\) é a parte inteira de \(x\).

\end{tcolorbox}

\subsection*{Exemplo 2}\label{exemplo-2}
\addcontentsline{toc}{subsection}{Exemplo 2}

Para simular valores de \(X\) com
\(P(X=j)=1/10, \, \,\, j=1,2,3,\ldots, 10\), você pode serguir o
procedimento a seguir:

\begin{longtable}[]{@{}
  >{\raggedright\arraybackslash}p{(\columnwidth - 10\tabcolsep) * \real{0.1333}}
  >{\raggedright\arraybackslash}p{(\columnwidth - 10\tabcolsep) * \real{0.1333}}
  >{\raggedright\arraybackslash}p{(\columnwidth - 10\tabcolsep) * \real{0.1333}}
  >{\raggedright\arraybackslash}p{(\columnwidth - 10\tabcolsep) * \real{0.1333}}
  >{\raggedright\arraybackslash}p{(\columnwidth - 10\tabcolsep) * \real{0.1733}}
  >{\raggedright\arraybackslash}p{(\columnwidth - 10\tabcolsep) * \real{0.2933}}@{}}
\toprule\noalign{}
\begin{minipage}[b]{\linewidth}\raggedright
\(j-1\)
\end{minipage} & \begin{minipage}[b]{\linewidth}\raggedright
\(j\)
\end{minipage} & \begin{minipage}[b]{\linewidth}\raggedright
\(U\)
\end{minipage} & \begin{minipage}[b]{\linewidth}\raggedright
\(kU\)
\end{minipage} & \begin{minipage}[b]{\linewidth}\raggedright
{\(X=j\)}
\end{minipage} & \begin{minipage}[b]{\linewidth}\raggedright
{\(X=\mathrm{Int}(kU)+1\)}
\end{minipage} \\
\midrule\noalign{}
\endhead
\bottomrule\noalign{}
\endlastfoot
0 & 1 & 0,01 & 0,1 & 1 & 1 \\
1 & 2 & 0,31 & 3,1 & 4 & 4 \\
2 & 3 & 0,53 & 5,3 & 6 & 6 \\
3 & 4 & 0,92 & 9,2 & 10 & 10 \\
4 & 5 & 0,45 & 4,5 & 5 & 5 \\
\(\vdots\) & \(\vdots\) & \(\vdots\) & \(\vdots\) & \(\vdots\) &
\(\vdots\) \\
9 & 10 & 0,74 & 7,4 & 8 & 8 \\
\end{longtable}

\subsubsection*{Exercícios}\label{exercuxedcios-1}
\addcontentsline{toc}{subsubsection}{Exercícios}

\begin{enumerate}
\def\labelenumi{\arabic{enumi}.}
\item
  Gerar uma permutação dos números \(1,2,3, \ldots, n\), considerando
  todas as \(n!\) possíveis permutações igualmente prováveis (exemplo
  4b).
\item
  Gerar valores de \(X\sim \mbox{Geométrica}(p)\) (exemplo 4d).
\item
  Implementar o algoritmo para gerar valores de
  \(X\sim \mbox{Poisson}(\lambda)\) (seção 4.2).
\item
  Implementar o algoritmo para gerar valores de
  \(X\sim \mbox{Binomial}(n,p)\) (seção 4.2).
\end{enumerate}

\subsection*{Exemplo 3}\label{exemplo-3}
\addcontentsline{toc}{subsection}{Exemplo 3}

Nosso objetivo agora é gerar valores de
\(X\sim \mbox{Poisson}(\lambda)\) com:

\[p_i=P(X=i)=\dfrac{e^{-\lambda}\lambda^i}{i!}, \quad i=0,1,2,\ldots. \]
::: \{.callout-important\} \#\# 🔎 \textbf{Identidade importante:}

\[p_{i+1} = \dfrac{\lambda}{i+1}p_i, \quad i\geq 0.\] :::

Uma forma bastante utilizada para gerar valores de uma variável
aleatória \(X\sim \mbox{Poisson}(\lambda)\) é por meio de algoritmos de
simulação baseados na identidade acima.

\begin{tcolorbox}[enhanced jigsaw, toprule=.15mm, arc=.35mm, colframe=quarto-callout-warning-color-frame, breakable, opacityback=0, rightrule=.15mm, bottomrule=.15mm, left=2mm, leftrule=.75mm, colback=white]

\vspace{-3mm}\textbf{Algoritmo 4}\vspace{3mm}

Passo 1. Gerar um número aleatório \(u \sim U(0,1)\).

Passo 2. Inicializar \(i = 0\), \(p = e^{-\lambda}\), \(F = p\).

Passo 3. Se \(u < F\), então definir \(X = i\) e parar.

Passo 4. Caso contrário:

\begin{itemize}
\item
  atualizar \(i \leftarrow i+1\),
\item
  atualizar \(p \leftarrow \frac{\lambda}{i}\,p\),
\item
  atualizar \(F \leftarrow F + p\),
\item
  voltar ao passo 3.
\end{itemize}

\end{tcolorbox}

\subsubsection*{Exercício}\label{exercuxedcio}
\addcontentsline{toc}{subsubsection}{Exercício}

Complete o código \texttt{R} abaixo:

\begin{Shaded}
\begin{Highlighting}[]
\NormalTok{N }\OtherTok{=} \DecValTok{10}\SpecialCharTok{\^{}}\DecValTok{5} \CommentTok{\# tamanho da amostra}
\NormalTok{L }\OtherTok{=} \DecValTok{3}  \CommentTok{\# lambda}
\NormalTok{x }\OtherTok{=} \FunctionTok{c}\NormalTok{()}
\ControlFlowTok{for}\NormalTok{ (j }\ControlFlowTok{in} \DecValTok{1}\SpecialCharTok{:}\NormalTok{N)\{}
\NormalTok{  u }\OtherTok{=} \FunctionTok{runif}\NormalTok{(}\DecValTok{1}\NormalTok{)}
\NormalTok{  i }\OtherTok{=} \DecValTok{0}\NormalTok{; p }\OtherTok{=} \FunctionTok{exp}\NormalTok{(}\SpecialCharTok{{-}}\NormalTok{L); F }\OtherTok{=}\NormalTok{ p}
\NormalTok{  aceito }\OtherTok{=} \StringTok{"não"}
  \ControlFlowTok{while}\NormalTok{ (aceito }\SpecialCharTok{!=} \StringTok{"sim"}\NormalTok{)\{}
    \ControlFlowTok{if}\NormalTok{(u }\SpecialCharTok{\textless{}}\NormalTok{ F) \{}
\NormalTok{       ...}
\NormalTok{     \} }\ControlFlowTok{else}\NormalTok{ \{}
\NormalTok{        ...}
\NormalTok{      \}}
\NormalTok{  \}}
\NormalTok{\}}
\end{Highlighting}
\end{Shaded}

\subsection*{Exemplo 4}\label{exemplo-4}
\addcontentsline{toc}{subsection}{Exemplo 4}

Por fim, iremos gerar valores da variável aleatório \(X\),
\(X\sim \mbox{Binomial}(n,p)\), com f.m.p dada por:

\[P(X=i)=\dfrac{n!}{i!(n-i)!}p^i(1-p)^{n-i}, \quad i=0,1,2,\ldots,n.\]

\begin{tcolorbox}[enhanced jigsaw, coltitle=black, bottomtitle=1mm, toprule=.15mm, arc=.35mm, colframe=quarto-callout-important-color-frame, breakable, opacityback=0, bottomrule=.15mm, rightrule=.15mm, titlerule=0mm, toptitle=1mm, title={🔎 \textbf{Identidade importante:}:}, leftrule=.75mm, opacitybacktitle=0.6, left=2mm, colback=white, colbacktitle=quarto-callout-important-color!10!white]

\[P(X=i+1) = \dfrac{n-i}{i+1}\dfrac{p}{1-p}P(X=i).\]

\end{tcolorbox}

A partir do resultado acima e do método da transformação inversa podemos
escrever o seguinte algoritmo:

\begin{tcolorbox}[enhanced jigsaw, toprule=.15mm, arc=.35mm, colframe=quarto-callout-warning-color-frame, breakable, opacityback=0, rightrule=.15mm, bottomrule=.15mm, left=2mm, leftrule=.75mm, colback=white]

\vspace{-3mm}\textbf{Algoritmo 5}\vspace{3mm}

Passo 1. Gerar um número aleatório \(u \sim U(0,1)\).

Passo 2. Calcular \(k = \dfrac{p}{1-p}\), inicializar \(i = 0\),
\(p_r = (1-p)^n\), \(F = p_r\).

Passo 3. Se \(u < F\), definir \(X = i\) e parar.

Passo 4. Caso contrário:

\begin{itemize}
\item
  atualizar \(p_r \leftarrow \dfrac{n-i}{i+1}\,k\,p_r\),
\item
  atualizar \(F \leftarrow F + p_r\),
\item
  atualizar \(i \leftarrow i+1\),
\item
  voltar ao passo 3.
\end{itemize}

\end{tcolorbox}

\subsection*{Exercício 5}\label{exercuxedcio-5}
\addcontentsline{toc}{subsection}{Exercício 5}

Escreve um código na linguagem \texttt{R} baseado no Algoritmo 5.
Utilize alguma ferramenta gráfica para verificar a coerência dos
resultados.

\section{Método da
Aceitação-Rejeição}\label{muxe9todo-da-aceitauxe7uxe3o-rejeiuxe7uxe3o}

Suponha que haja interesse em simular valores de uma v.a. \(X\) e que
não seja possível inverter a sua função de distribuição ou não dispomos
de um método para gerar dessa variável aleatória. Entretanto, sabemos
como simular de uma outra v.a. \(Y\) e que é possível estabelecer uma
relação entre as probabilidades associadas às duas variáveis aleatórias
(\(X\) e \(Y\)) de tal modo que valores de \(Y\) possam ser admitidos
como valores de \(X\).

\begin{longtable}[]{@{}
  >{\raggedright\arraybackslash}p{(\columnwidth - 20\tabcolsep) * \real{0.1045}}
  >{\raggedright\arraybackslash}p{(\columnwidth - 20\tabcolsep) * \real{0.0896}}
  >{\raggedright\arraybackslash}p{(\columnwidth - 20\tabcolsep) * \real{0.0896}}
  >{\raggedright\arraybackslash}p{(\columnwidth - 20\tabcolsep) * \real{0.0896}}
  >{\raggedright\arraybackslash}p{(\columnwidth - 20\tabcolsep) * \real{0.0896}}
  >{\raggedright\arraybackslash}p{(\columnwidth - 20\tabcolsep) * \real{0.0896}}
  >{\raggedright\arraybackslash}p{(\columnwidth - 20\tabcolsep) * \real{0.0896}}
  >{\raggedright\arraybackslash}p{(\columnwidth - 20\tabcolsep) * \real{0.0896}}
  >{\raggedright\arraybackslash}p{(\columnwidth - 20\tabcolsep) * \real{0.0896}}
  >{\raggedright\arraybackslash}p{(\columnwidth - 20\tabcolsep) * \real{0.0896}}
  >{\raggedright\arraybackslash}p{(\columnwidth - 20\tabcolsep) * \real{0.0896}}@{}}
\toprule\noalign{}
\begin{minipage}[b]{\linewidth}\raggedright
\(x\)
\end{minipage} & \begin{minipage}[b]{\linewidth}\raggedright
1
\end{minipage} & \begin{minipage}[b]{\linewidth}\raggedright
2
\end{minipage} & \begin{minipage}[b]{\linewidth}\raggedright
3
\end{minipage} & \begin{minipage}[b]{\linewidth}\raggedright
4
\end{minipage} & \begin{minipage}[b]{\linewidth}\raggedright
5
\end{minipage} & \begin{minipage}[b]{\linewidth}\raggedright
6
\end{minipage} & \begin{minipage}[b]{\linewidth}\raggedright
7
\end{minipage} & \begin{minipage}[b]{\linewidth}\raggedright
8
\end{minipage} & \begin{minipage}[b]{\linewidth}\raggedright
9
\end{minipage} & \begin{minipage}[b]{\linewidth}\raggedright
10
\end{minipage} \\
\midrule\noalign{}
\endhead
\bottomrule\noalign{}
\endlastfoot
\(p_x\) & 0,11 & 0,12 & 0,09 & 0,08 & 0,12 & 0,10 & 0,09 & 0,09 & 0,10 &
0,10 \\
\end{longtable}

\begin{longtable}[]{@{}
  >{\raggedright\arraybackslash}p{(\columnwidth - 20\tabcolsep) * \real{0.1045}}
  >{\raggedright\arraybackslash}p{(\columnwidth - 20\tabcolsep) * \real{0.0896}}
  >{\raggedright\arraybackslash}p{(\columnwidth - 20\tabcolsep) * \real{0.0896}}
  >{\raggedright\arraybackslash}p{(\columnwidth - 20\tabcolsep) * \real{0.0896}}
  >{\raggedright\arraybackslash}p{(\columnwidth - 20\tabcolsep) * \real{0.0896}}
  >{\raggedright\arraybackslash}p{(\columnwidth - 20\tabcolsep) * \real{0.0896}}
  >{\raggedright\arraybackslash}p{(\columnwidth - 20\tabcolsep) * \real{0.0896}}
  >{\raggedright\arraybackslash}p{(\columnwidth - 20\tabcolsep) * \real{0.0896}}
  >{\raggedright\arraybackslash}p{(\columnwidth - 20\tabcolsep) * \real{0.0896}}
  >{\raggedright\arraybackslash}p{(\columnwidth - 20\tabcolsep) * \real{0.0896}}
  >{\raggedright\arraybackslash}p{(\columnwidth - 20\tabcolsep) * \real{0.0896}}@{}}
\toprule\noalign{}
\begin{minipage}[b]{\linewidth}\raggedright
\(y\)
\end{minipage} & \begin{minipage}[b]{\linewidth}\raggedright
1
\end{minipage} & \begin{minipage}[b]{\linewidth}\raggedright
2
\end{minipage} & \begin{minipage}[b]{\linewidth}\raggedright
3
\end{minipage} & \begin{minipage}[b]{\linewidth}\raggedright
4
\end{minipage} & \begin{minipage}[b]{\linewidth}\raggedright
5
\end{minipage} & \begin{minipage}[b]{\linewidth}\raggedright
6
\end{minipage} & \begin{minipage}[b]{\linewidth}\raggedright
7
\end{minipage} & \begin{minipage}[b]{\linewidth}\raggedright
8
\end{minipage} & \begin{minipage}[b]{\linewidth}\raggedright
9
\end{minipage} & \begin{minipage}[b]{\linewidth}\raggedright
10
\end{minipage} \\
\midrule\noalign{}
\endhead
\bottomrule\noalign{}
\endlastfoot
\(q_y\) & 0,10 & 0,10 & 0,10 & 0,10 & 0,10 & 0,10 & 0,10 & 0,10 & 0,10 &
0,10 \\
\end{longtable}

Contexto:

\begin{itemize}
\tightlist
\item
  \textbf{Podemos}: simular valores de \(Y\) com f.p.
  \(q_j=P(Y=j), j\geq 0\).
\item
  \textbf{Queremos}: simular valores de \(X\) com f.p.
  \small  \(p_j=P(X=j), j\geq 0\).
\item
  \textbf{Proposta}: simular valor \(y\) de \(Y\) e aceitar esse valor
  para \(X\) com probabilidade proporcional a \(\dfrac{p_y}{q_y}\).
\end{itemize}

\begin{tcolorbox}[enhanced jigsaw, toprule=.15mm, arc=.35mm, colframe=quarto-callout-warning-color-frame, breakable, opacityback=0, rightrule=.15mm, bottomrule=.15mm, left=2mm, leftrule=.75mm, colback=white]

\vspace{-3mm}\textbf{Algoritmo}\vspace{3mm}

Passo 1. Simular \(y\) de \(Y\) com f.m.p. \(q_y\).

Passo 2. Gerar \(u \sim U(0,1)\).

Passo 3. Se \(u < \dfrac{p_y}{c \, q_y}\), então aceite \(X = y\). Caso
contrário, rejeite e não atribua valor a \(X\).

Passo 4. Repetir os passos 1--3 até obter o tamanho de amostra desejado.

\end{tcolorbox}

O algoritmo aceitação-rejeição gera uma v.a. \(X\) tal que
\(P(X=j)=p_j\), \(j=0,1,2,\ldots\). Além disso, o número de iterações
que o algoritmo necessita para obter \(X\) é uma v.a. geométrica com
média \(c\).

\begin{tcolorbox}[enhanced jigsaw, toprule=.15mm, arc=.35mm, colframe=quarto-callout-note-color-frame, breakable, opacityback=0, rightrule=.15mm, bottomrule=.15mm, left=2mm, leftrule=.75mm, colback=white]

\vspace{-3mm}\textbf{Prova do Teorema}\vspace{3mm}

\begin{enumerate}
\def\labelenumi{\arabic{enumi}.}
\tightlist
\item
  Em uma iteração, determinar a probabilidade de gerar e ser aceito o
  valor \(j\):
  \[P(Y=j,aceitar)= P(Y=y).P(aceitar/Y=j)= q_j.\dfrac{pj}{cq_j} = \dfrac{p_j}{c}\]
\item
  Calcular a probabilidade de aceitar um valor \(j\) gerado:
  \[P(aceitar)= \sum_j P(Y=j,aceitar)=\sum_j \dfrac{pj}{c} = \dfrac{1}{c}\]
\item
  Como cada iteração independentemente resulta um valor aceitável com
  probabilidade \(\frac{1}{c}\), o número de iterações necessárias segue
  uma geométrica de média \(c\). Portanto,
  \[P(X=j)=\! \sum_n P(\mbox{\textit{j aceito na iteração n}}) = \sum_n \left(\!1\!-\!\dfrac{1}{c}\!\right)^{\!\!n-1}\!\!.\dfrac{p_j}{c}=p_j\].
\end{enumerate}

\end{tcolorbox}

\begin{tcolorbox}[enhanced jigsaw, coltitle=black, bottomtitle=1mm, toprule=.15mm, arc=.35mm, colframe=quarto-callout-important-color-frame, breakable, opacityback=0, bottomrule=.15mm, rightrule=.15mm, titlerule=0mm, toptitle=1mm, title={Observações}, leftrule=.75mm, opacitybacktitle=0.6, left=2mm, colback=white, colbacktitle=quarto-callout-important-color!10!white]

\begin{itemize}
\item
  A constante \(c\) está relacionada com o número de interações
  necessárias até a aceitação de um valor de \(Y\) para \(X\).
\item
  O valor \(c\) deve ser o menor possível.
\item
  O valor de \(c\) será o \(\mbox{max}\left\{\dfrac{p_y}{q_y}\right\}\).
\end{itemize}

\end{tcolorbox}

\begin{tcolorbox}[enhanced jigsaw, coltitle=black, bottomtitle=1mm, toprule=.15mm, arc=.35mm, colframe=quarto-callout-note-color-frame, breakable, opacityback=0, bottomrule=.15mm, rightrule=.15mm, titlerule=0mm, toptitle=1mm, title=\textcolor{quarto-callout-note-color}{\faInfo}\hspace{0.5em}{Nota}, leftrule=.75mm, opacitybacktitle=0.6, left=2mm, colback=white, colbacktitle=quarto-callout-note-color!10!white]

\[u<\dfrac{p_y}{c.q_y} \ \ \ \ \mbox{e}  \ \ \ \  P(U<\dfrac{p_y}{c.q_y})= \dfrac{p_y}{c.q_y} \]
\[\Downarrow\]
\[\dfrac{p_y}{c.q_y}\leq 1 \ \ \ \ \mbox{para todo} \ \ y \ \ \mbox{tal que} \ \ p_y>0\]
\[\Downarrow\] \[c=\mbox{max}\left\{\dfrac{p_y}{q_y}\right\}\]

\end{tcolorbox}

\subsubsection*{Exemplo 1}\label{exemplo-1-1}
\addcontentsline{toc}{subsubsection}{Exemplo 1}

Gerar um valor da variável aleatória \(X\) com f.m.p.:

\begin{longtable}[]{@{}
  >{\raggedright\arraybackslash}p{(\columnwidth - 20\tabcolsep) * \real{0.1045}}
  >{\raggedright\arraybackslash}p{(\columnwidth - 20\tabcolsep) * \real{0.0896}}
  >{\raggedright\arraybackslash}p{(\columnwidth - 20\tabcolsep) * \real{0.0896}}
  >{\raggedright\arraybackslash}p{(\columnwidth - 20\tabcolsep) * \real{0.0896}}
  >{\raggedright\arraybackslash}p{(\columnwidth - 20\tabcolsep) * \real{0.0896}}
  >{\raggedright\arraybackslash}p{(\columnwidth - 20\tabcolsep) * \real{0.0896}}
  >{\raggedright\arraybackslash}p{(\columnwidth - 20\tabcolsep) * \real{0.0896}}
  >{\raggedright\arraybackslash}p{(\columnwidth - 20\tabcolsep) * \real{0.0896}}
  >{\raggedright\arraybackslash}p{(\columnwidth - 20\tabcolsep) * \real{0.0896}}
  >{\raggedright\arraybackslash}p{(\columnwidth - 20\tabcolsep) * \real{0.0896}}
  >{\raggedright\arraybackslash}p{(\columnwidth - 20\tabcolsep) * \real{0.0896}}@{}}
\toprule\noalign{}
\begin{minipage}[b]{\linewidth}\raggedright
\(j\)
\end{minipage} & \begin{minipage}[b]{\linewidth}\raggedright
1
\end{minipage} & \begin{minipage}[b]{\linewidth}\raggedright
2
\end{minipage} & \begin{minipage}[b]{\linewidth}\raggedright
3
\end{minipage} & \begin{minipage}[b]{\linewidth}\raggedright
4
\end{minipage} & \begin{minipage}[b]{\linewidth}\raggedright
5
\end{minipage} & \begin{minipage}[b]{\linewidth}\raggedright
6
\end{minipage} & \begin{minipage}[b]{\linewidth}\raggedright
7
\end{minipage} & \begin{minipage}[b]{\linewidth}\raggedright
8
\end{minipage} & \begin{minipage}[b]{\linewidth}\raggedright
9
\end{minipage} & \begin{minipage}[b]{\linewidth}\raggedright
10
\end{minipage} \\
\midrule\noalign{}
\endhead
\bottomrule\noalign{}
\endlastfoot
\(p_j\) & 0,11 & 0,12 & 0,09 & 0,08 & 0,12 & 0,10 & 0,09 & 0,09 & 0,10 &
0,10 \\
\end{longtable}

Considerando que sabemos gerar de uma v.a. uniforme discreta,
assumiremos \(Y\) com distribuição

\[P(Y=j)=q_j=\dfrac{1}{10}; \quad j=1,2, \ldots, 10.\]

A constante \(c\) será determinada por
\(c=\mbox{max}\left\{\dfrac{p_j}{q_j}\right\} =\dfrac{0,\!12}{0,\!10}=1,\!2.\)

\begin{tcolorbox}[enhanced jigsaw, toprule=.15mm, arc=.35mm, colframe=quarto-callout-warning-color-frame, breakable, opacityback=0, rightrule=.15mm, bottomrule=.15mm, left=2mm, leftrule=.75mm, colback=white]

\vspace{-3mm}\textbf{Algoritmo}\vspace{3mm}

\begin{enumerate}
\def\labelenumi{\arabic{enumi}.}
\tightlist
\item
  Simular \(y\) de \(Y\): gere \(u_1 \sim U(0,1)\) e faça
  \(y = \mathrm{Int}(10u_1)+1\).
\item
  Gerar um segundo número aleatório \(u_2\).
\item
  Se \(u_2 < \dfrac{p_y}{0.12}\), faça \(X=y\) e pare. Caso contrário,
  retorne ao passo 1.
\end{enumerate}

\end{tcolorbox}

\begin{tcolorbox}[enhanced jigsaw, coltitle=black, bottomtitle=1mm, toprule=.15mm, arc=.35mm, colframe=quarto-callout-note-color-frame, breakable, opacityback=0, bottomrule=.15mm, rightrule=.15mm, titlerule=0mm, toptitle=1mm, title=\textcolor{quarto-callout-note-color}{\faInfo}\hspace{0.5em}{Nota}, leftrule=.75mm, opacitybacktitle=0.6, left=2mm, colback=white, colbacktitle=quarto-callout-note-color!10!white]

Suponha \(y=1\). Então, se
\(u_2<\dfrac{p_1}{0,\!12}=\dfrac{0,\!11}{0,\!12}=0,\!9167\), faremos
\(X=1\). Isto é, assumiremos que o valor 1 gerado é plausível de ser da
distribuição de \(X\). O gráfico a seguir ilustra esse processo.

\end{tcolorbox}

\begin{Shaded}
\begin{Highlighting}[]
\CommentTok{\#set.seed(20252)}
\NormalTok{pseudo\_x }\OtherTok{\textless{}{-}} \ConstantTok{NULL}
\ControlFlowTok{for}\NormalTok{(i }\ControlFlowTok{in} \FunctionTok{seq\_len}\NormalTok{(}\DecValTok{1000}\NormalTok{))\{}
  
\NormalTok{pj }\OtherTok{\textless{}{-}} \FunctionTok{c}\NormalTok{(}\FloatTok{0.11}\NormalTok{, }\FloatTok{0.12}\NormalTok{, }\FloatTok{0.09}\NormalTok{, }\FloatTok{0.08}\NormalTok{, }\FloatTok{0.12}\NormalTok{, }\FloatTok{0.10}\NormalTok{, }\FloatTok{0.09}\NormalTok{, }\FloatTok{0.09}\NormalTok{, }\FloatTok{0.10}\NormalTok{, }\FloatTok{0.10}\NormalTok{)}

\NormalTok{u1 }\OtherTok{\textless{}{-}} \FunctionTok{runif}\NormalTok{(}\DecValTok{1}\NormalTok{)}
\NormalTok{y }\OtherTok{\textless{}{-}} \FunctionTok{floor}\NormalTok{(}\DecValTok{10}\SpecialCharTok{*}\NormalTok{u1) }\SpecialCharTok{+} \DecValTok{1}


\ControlFlowTok{repeat}\NormalTok{\{}
\NormalTok{u2 }\OtherTok{\textless{}{-}} \FunctionTok{runif}\NormalTok{(}\DecValTok{1}\NormalTok{)}
\NormalTok{x }\OtherTok{\textless{}{-}}\NormalTok{ y}

\ControlFlowTok{if}\NormalTok{(u2 }\SpecialCharTok{\textless{}}\NormalTok{ pj[y]}\SpecialCharTok{/}\FloatTok{0.12}\NormalTok{) }\ControlFlowTok{break}
\NormalTok{\}}

\NormalTok{pseudo\_x[i] }\OtherTok{\textless{}{-}}\NormalTok{ x}
\NormalTok{\}}

\FunctionTok{round}\NormalTok{(}\FunctionTok{prop.table}\NormalTok{(}\FunctionTok{table}\NormalTok{(pseudo\_x)),}\DecValTok{2}\NormalTok{)}
\end{Highlighting}
\end{Shaded}

\begin{verbatim}
pseudo_x
   1    2    3    4    5    6    7    8    9   10 
0.12 0.10 0.09 0.09 0.10 0.09 0.09 0.11 0.11 0.10 
\end{verbatim}

Existe algum problema com os resultados acima? Se sim, faça a correção
do código e verifique graficamente a coerência dos resultados.

\begin{tcolorbox}[enhanced jigsaw, coltitle=black, bottomtitle=1mm, toprule=.15mm, arc=.35mm, colframe=quarto-callout-important-color-frame, breakable, opacityback=0, bottomrule=.15mm, rightrule=.15mm, titlerule=0mm, toptitle=1mm, title={Observações}, leftrule=.75mm, opacitybacktitle=0.6, left=2mm, colback=white, colbacktitle=quarto-callout-important-color!10!white]

\begin{itemize}
\item
  \(91,\!67\%\) de todos os valores 1 gerados de \(Y\) serão aceitos
\item
  \(100\%\) dos valores 2 gerados de \(Y\) serão aceitos
\item
  \(75\%\) dos valores 3 gerados de \(Y\) serão aceitos
\item
  Qualquer valor da constante \(c\) inferior a \(1,\!2\) impossibilita
  gerar a distribuição de \(X\)
\item
  Qualquer valor da constante \(c\) superior a \(1,\!2\) tornaria o
  processo mais lento para a obteção da amostra
\end{itemize}

\end{tcolorbox}

\subsubsection*{Exercícios}\label{exercuxedcios-2}
\addcontentsline{toc}{subsubsection}{Exercícios}

\begin{enumerate}
\def\labelenumi{\arabic{enumi}.}
\item
  Gere números pseudoaleatórios de \(X\) considerando \(c = 2,4\).
\item
  Compare com os resultados obtidos no Exemplo.
\end{enumerate}

\section{Método da Composição}\label{muxe9todo-da-composiuxe7uxe3o}

Suponha que tenhamos um método eficiente para simular o valor de uma
variável aleatória com f.m.p. \({p_j, ; j \geq 0}\) ou
\({q_j, ; j \geq 0}\), e que desejamos simular a variável aleatória
\(X\) com f.m.p.:

\[\Pr(X = j) = \alpha p_j + (1 - \alpha)q_j, \quad j \geq 0, 0 < a < 1.\]
Se \(X_1 \sim {p_j}\) e \(X_2 \sim {q_j}\), então podemos definir

\[X =
\begin{cases}
X_1, & \text{com probabilidade } a, \\
X_2, & \text{com probabilidade } 1-a ,
\end{cases}.\]

Assim, \(X\) terá exatamente a função de probabilidade acima.

\subsubsection*{Exemplo}\label{exemplo-5}
\addcontentsline{toc}{subsubsection}{Exemplo}

Suponha que desejamos gerar o valor de uma variável aleatória \(X\) tal
que

\[p_j = \Pr(X = j) =  \begin{cases}
0.05, & \text{para}\quad j = 1, 2, 3, 4, 5, \\
0.15, & \text{para}\quad  j = 6, 7, 8, 9, 10,
\end{cases}\]

Note que \(p_j = 0.5 \times p_j^{(1)} + 0.5 \times p_j^{(2)},\) em que

\[p_j^{(1)} = 0.1, \quad j = 1, \dots, 10 \quad \text{e} \quad p_j^{(2)} = \begin{cases}
0, & \text{para}\quad j = 1, 2, 3, 4, 5, \\
0.2, & \text{para}\quad  j = 6, 7, 8, 9, 10,
\end{cases}\]

podemos realizar essa simulação gerando primeiro um número aleatório
\(U \sim U(0,1)\) e então:

\begin{itemize}
\tightlist
\item
  Se \(U < 0.5\), gerar \(X\) de uma uniforme discreta sobre
  \(\{1,2,\dots,10\}\).
\item
  Caso contrário (\(U \geq 0.5\)), gerar \(X\) de uma uniforme discreta
  sobre \(\{6,7,8,9,10\}\).
\end{itemize}

\begin{tcolorbox}[enhanced jigsaw, toprule=.15mm, arc=.35mm, colframe=quarto-callout-warning-color-frame, breakable, opacityback=0, rightrule=.15mm, bottomrule=.15mm, left=2mm, leftrule=.75mm, colback=white]

\vspace{-3mm}\textbf{Algoritmo}\vspace{3mm}

Passo 1. Gerar \(U_1 \sim U(0,1)\).

Passo 2. Gerar \(U_2 \sim U(0,1)\).

Passo 3. Se \(U_1 < 0.5\), definir \(X = \mathrm{Int}(10U_1)+1\). Caso
contrário, definir \(X = \mathrm{Int}(5U_2)+6\).

\end{tcolorbox}

Se \(F_i\), \(i=1,\dots,n\) são funções de distribuição e \(\alpha_i\),
\(i=1,\dots,n\) são números não negativos cuja soma é 1, então a função
de distribuição:

\[
F(x) = \sum_{i=1}^n \alpha_i F_i(x),
\]

é uma \textbf{mistura}, ou uma \textbf{composição}, das funções de
distribuição \(F_i\), \(i = 1, \dots, n\).

Uma maneira de simular a partir de \(F\) é primeiro simular uma variável
aleatória \(I\), igual a \(i\) com probabilidade \(\alpha_i\),
\(i=1,\dots,n\), e então simular a partir da distribuição \(F_I\) (Isto
é, se o valor simulado de \(I\) for \(I = j\), então a segunda simulação
é feita a partir de \(F_j\)). Essa abordagem para simular de \(F\) é
frequentemente chamada de \textbf{método de composição}.

\bookmarksetup{startatroot}

\chapter{Otimização Numérica}\label{otimizauxe7uxe3o-numuxe9rica}

\section{Método de Newton}\label{muxe9todo-de-newton}

\section{Método de Newton-Raphson}\label{muxe9todo-de-newton-raphson}

\section{Método Escore de Fisher}\label{muxe9todo-escore-de-fisher}

\section{Método BFGS}\label{muxe9todo-bfgs}

\bookmarksetup{startatroot}

\chapter{Métodos de Reamostragem}\label{muxe9todos-de-reamostragem}

\section{Bootstrap}\label{bootstrap}

\subsection{Introdução}\label{introduuxe7uxe3o-2}

Bootstrap é um método (computacional) de reamostragem baseado em
subamostras de uma amostra observada, sendo introduzido por Efron
(1979). Pode ser utilizado com o propósito de estimar erros padrão, viés
de estimadores, construir intervalos de confiança, testes de hipóteses,
entre outros. Pode ser utilizando sob duas abordagens:
\textbf{paramétrica} e \textbf{não paramétrica}. A abordagem paramétrica
exige um modelo estatística, enquanto que na abordagem não paramétrica
não há suposição de modelo estatístico; toma-se por base uma
distribuição empírica que atribui probabilidade \(1/n\) para cada um dos
\(n\) elementos da amostra. Algumas referências importantes neste tema
são: Efron (1979), Wu (1986), Fisher \(\&\) Hall (1989), Fredman (1986),
Efron \(\&\) Tibshirani (1993), Horowitz (1997), Davison \(\&\) Hinkley
(1997).

\subsection{Acurária da média
amostral}\label{acuruxe1ria-da-muxe9dia-amostral}

Experimento com 16 ratos, divididos em dois grupos: um grupo recebeu o
tratamento e um outro grupo não recebeu o tratamento (controle). O tempo
de sobrevivência (dias) é apresentado para cada um dos ratos. O
tratamento prolonga a vida?

\begin{longtable}[]{@{}
  >{\raggedright\arraybackslash}p{(\columnwidth - 12\tabcolsep) * \real{0.2025}}
  >{\raggedright\arraybackslash}p{(\columnwidth - 12\tabcolsep) * \real{0.1139}}
  >{\raggedright\arraybackslash}p{(\columnwidth - 12\tabcolsep) * \real{0.1139}}
  >{\raggedright\arraybackslash}p{(\columnwidth - 12\tabcolsep) * \real{0.1139}}
  >{\raggedright\arraybackslash}p{(\columnwidth - 12\tabcolsep) * \real{0.0506}}
  >{\raggedright\arraybackslash}p{(\columnwidth - 12\tabcolsep) * \real{0.1013}}
  >{\raggedright\arraybackslash}p{(\columnwidth - 12\tabcolsep) * \real{0.3038}}@{}}
\toprule\noalign{}
\begin{minipage}[b]{\linewidth}\raggedright
grupo
\end{minipage} & \begin{minipage}[b]{\linewidth}\raggedright
tempo 1
\end{minipage} & \begin{minipage}[b]{\linewidth}\raggedright
tempo 2
\end{minipage} & \begin{minipage}[b]{\linewidth}\raggedright
tempo 3
\end{minipage} & \begin{minipage}[b]{\linewidth}\raggedright
n
\end{minipage} & \begin{minipage}[b]{\linewidth}\raggedright
média
\end{minipage} & \begin{minipage}[b]{\linewidth}\raggedright
\(\widehat{ep}\)(média)
\end{minipage} \\
\midrule\noalign{}
\endhead
\bottomrule\noalign{}
\endlastfoot
tratamento (X) & 94 & 197 & 16 & 7 & 86,86 & 25,24 \\
& 38 & 99 & 141 & & & \\
& 23 & & & & & \\
controle (Y) & 52 & 104 & 146 & 9 & 56,22 & 14,14 \\
& 10 & 51 & 30 & & & \\
& 40 & 27 & 46 & & & \\
diferença & & & & & 30,63 & 28,93 \\
\end{longtable}

Alguns pontos importantes são:

\begin{itemize}
\item
  A resposta à pergunta dependerá de quão acurado(s) é(são) o(s)
  estimador(es).
\item
  O erro padrão é uma medida (muito usual) de acurácia de estimador.
\item
  erro padrão estimado para a média amostral \(\bar{X}\)
  \[\widehat{ep}(\bar{X})=\sqrt{\dfrac{s^2}{n}},\] em que
  \(s^2=\sum_{i=1}^n(x_i-\bar{x})^2/(n-1)\).
\item
  erro padrão de qualquer estimador é definido pela raiz quadrada de sua
  variância
\end{itemize}

Além disso, temos que:

\begin{itemize}
\item
  erro padrão pequeno \(\longrightarrow\) acurácia alta
\item
  erro padrão grande \(\longrightarrow\) acurácia baixa
\item
  acurácia alta (baixa) indica que o estimador apresenta valores
  próximos (distantes) ao seu valor esperado
\item
  espera-se que 68\(\%\) dos valores do estimador estejam a menos de um
  erro padrão do seu valor esperado, e 95\(\%\), a menos de dois erros
  padrões
\item
  erro padrão da diferença \((\bar{X}-\bar{Y})\):
  \[28.93=\sqrt{25.24^2+14.14^2}.\]
\end{itemize}

\textbf{Resposta à pergunta:} a diferença observada 30.63 é somente
30.63/28.93=1.05 erros padrões (estimados) maior que zero, indicando um
resultado não significativo, ou seja, o tratamento não aumenta o tempo
médio de vida (considerando a teoria dos testes de hipóteses).

O erro padrão para o estimador média amostral apresenta fórmula
conhecida, mas há casos em que não dispomos de fórmulas. Suponha que
haja interesse em comparar os dois grupos de ratos em relação aos tempos
medianos. Temos: md(X)=94 e md(Y)=46. A diferença é 48, maior que a
diferença para as médias. Com base na mediana, o tratamento prolonga a
vida?

\subsection{Estimativa bootstrap do erro
padrão}\label{estimativa-bootstrap-do-erro-padruxe3o}

Considerações:

\begin{itemize}
\item
  \(\underset{\sim}{x} = (x_1,\,x_2\,\ldots,\,x_n)\): vetor de dados
  observado (amostra original de tamanho \(n\))
\item
  \(s(\underset{\sim}{x})\): estatística de interesse (por exemplo,
  média amostral)
\item
  Uma amostra bootstrap
  \(\underset{\sim}{x}^*=(x_1^*, x_2^*, \ldots, x_n^*)\) é obtida pela
  amostragem aleatória de tamanho \(n\), com reposição, de
  \(\underset{\sim}{x}\). Por exemplo, com \(n=7\), poderíamos obter
  \(\underset{\sim}{x}^*=(x_1^*, x_2^*, \ldots, x_n^*)=(x_5, x_7, x_5, x_4, x_7, x_3, x_1)\).
\end{itemize}

\textbf{O algoritmo bootstrap}

\begin{itemize}
\item
  gerar \(B\) amostras bootstrap independentes:
  \(\underset{\sim}{x}^{*^{1}}, \underset{\sim}{x}^{*^{2}}, \ldots, \underset{\sim}{x}^{*^{B}},\)
  cada uma de tamanho \(n\) e \(50 \leq B \leq 200\).
\item
  calcular \(s(\underset{\sim}{x}^{*^{b}})\), \(b=1, 2, \ldots, B\).
  \(s(\underset{\sim}{x}^{*^{b}})\) é denominada réplica bootstrap de
  \(s(\underset{\sim}{x})\)
\item
  calcular
  \(\hat{ep}_{boot}=\hat{ep}_{B}=\sqrt{ \dfrac{\sum_{b=1}^{B}[s(\underset{\sim}{x}^{*^{b}})-s(.)]^2}{B-1}},\)em
  que \(s(.)=\frac{\sum_{b=1}^{B}s(\underset{\sim}{x}^{*^{b}})}{B}\)
\end{itemize}

Sintaxe do \texttt{R} para calcular a estimativa bootstrap do erro
padrão da média do tempo de sobrevida dos ratos do grupo tratamento.

\begin{Shaded}
\begin{Highlighting}[]
\FunctionTok{set.seed}\NormalTok{(}\DecValTok{1234}\NormalTok{)}
\CommentTok{\# grupo tratamento (amostra original)}
\NormalTok{x }\OtherTok{\textless{}{-}} \FunctionTok{c}\NormalTok{(}\DecValTok{94}\NormalTok{, }\DecValTok{197}\NormalTok{, }\DecValTok{16}\NormalTok{, }\DecValTok{38}\NormalTok{, }\DecValTok{99}\NormalTok{, }\DecValTok{141}\NormalTok{, }\DecValTok{23}\NormalTok{) }
\NormalTok{n }\OtherTok{\textless{}{-}} \FunctionTok{length}\NormalTok{(x)}
\NormalTok{s }\OtherTok{\textless{}{-}} \DecValTok{0}  \CommentTok{\# estatística de interesse}
\NormalTok{B }\OtherTok{\textless{}{-}} \DecValTok{50} \CommentTok{\# no. de amostras bootstrap }

\ControlFlowTok{for}\NormalTok{(i }\ControlFlowTok{in} \DecValTok{1}\SpecialCharTok{:}\NormalTok{B)\{}
   \CommentTok{\# réplica bootstrap para o estimador média}
\NormalTok{   s[i] }\OtherTok{\textless{}{-}} \FunctionTok{mean}\NormalTok{(}\FunctionTok{sample}\NormalTok{(x,n,}\AttributeTok{replace=}\ConstantTok{TRUE}\NormalTok{)) }
\NormalTok{ \}}
\CommentTok{\# estimativa bootstrap para o erro padrão da média}
\NormalTok{ep }\OtherTok{\textless{}{-}}  \FunctionTok{sd}\NormalTok{(s); ep}
\end{Highlighting}
\end{Shaded}

\begin{verbatim}
[1] 27.41873
\end{verbatim}

Estimativas bootstrap do erro padrão da média e da mediana do tempo de
sobrevivência dos ratos do grupo tratamento. A mediana é menos acurada
(erros padrões maiores) que a média para esse conjunto de dados.

\begin{longtable}[]{@{}lllllll@{}}
\toprule\noalign{}
\(B\) & 50 & 100 & 250 & 500 & 1000 & \(\infty\) \\
\midrule\noalign{}
\endhead
\bottomrule\noalign{}
\endlastfoot
média & 19.72 & 23.63 & 22.32 & 23.79 & 23.02 & 23.36 \\
mediana & 32.21 & 36.35 & 34.46 & 36.72 & 36.48 & 37.83 \\
\end{longtable}

Formalização:

\begin{itemize}
\item
  \(\underset{\sim}{X} = X_1, X_2,\dots, X_n\): amostra aleatória de uma
  f.d.a. \(F\)
\item
  \(\underset{\sim}{x} = (x_1, x_2,\ldots, x_n)\): amostra aleatória
  observada de \(F\)
\item
  \(\Theta=t(F)\): parâmetro (\(\Theta\) uma função de \(F\))
\item
  \(\hat{\Theta}=s(\underset{\sim}{X})\): estimador para \(\Theta\)
\item
  \(\hat{\theta}=s(\underset{\sim}{x})\): estimativa para \(\Theta\)
\item
  Quão acurado é o estimador \(\hat{\Theta}\)?
\item
  Seja \(\hat{F}\) a distribuição empírica que atribui a probabilidade
  \(1/n\) para cada valor observado \(x_i\), \(i=1,2,\ldots, n\). A
  amostra bootstrap \(\underset{\sim}{x}^*\) é definida como a amostra
  aleatória com reposição de tamanho \(n\) extraída de \(\hat{F}\).
  \[\underset{\sim}{x}^*=(x_1^*,\,x_2^*\,\ldots,\,x_n^*)\]
  \[\hat{F}\longrightarrow (x_1^*,\,x_2^*\,\ldots,\,x_n^*)\]
  \textbf{Nota:}
\item
  \(\underset{\sim}{x}^*\): o símbolo \(*\) indica que a amostra não é a
  original \(\underset{\sim}{x}\), mas uma versão
  aleatorizada(reamostrada) de \(\underset{\sim}{x}\)
\item
  a cada amostra bootstrap corresponde uma réplica bootstrap de
  \(\hat\theta\), \(\hat\theta^* =s(\underset{\sim}{x}^*)\)
\item
  \(ep_F(\hat\Theta)\) é estimado por \(ep_{\hat{F}}(\hat\Theta^*)\),
  chamado \textbf{estimador bootstrap ideal} para o erro padrão
  \(\hat\Theta\)
\item
  Raramente faz-se necessário \(B\geq 200\) para estimar erro padrão;
  valores (muito) maiores são necessários, por exemplo, para IC
  bootstrap.
\item
  \(\displaystyle\lim_{B \to \infty} \hat{ep}_B=ep_{\hat{F}}=ep_{\hat{F}}(\hat\Theta^*)\)
\item
  O estimador bootstrap ideal \(ep_{\hat{F}}(\hat\Theta^*)\) e sua
  aproximação \(\hat{ep}_B\) são chamados \textbf{estimadores bootstrap
  não-paramétricos}, pois baseiam-se em \(\hat{F}\), o estimador
  não-paramétrico de \(F\).
\item
  total de amostras bootstrap distintas (combinação com repetição):
  \[\binom{2n -1}{n}.\]
\item
  No \texttt{R}: ver as funções \texttt{factorial()}, \texttt{choose()},
  \texttt{combn()}, \texttt{combinations()}.
\end{itemize}

\subsubsection{Exemplo}\label{exemplo-6}

Dados de faculdades americanas de direito. População: \(N=82\)
faculdades. Amostra aleatória: \(n=15\) faculdades. Variáveis
analisadas: LSAT (escore médio em um teste), GPA (pontuação média na
faculdade).

\begin{longtable}[]{@{}llllll@{}}
\toprule\noalign{}
escola & LSAT & GPA & escola & LSAT & GPA \\
\midrule\noalign{}
\endhead
\bottomrule\noalign{}
\endlastfoot
1 & 576 & 3,39 & 9 & 651 & 3,36 \\
2 & 635 & 3,30 & 10 & 605 & 3,13 \\
3 & 558 & 2,81 & 11 & 653 & 3,12 \\
4 & 578 & 3,03 & 12 & 575 & 2,74 \\
5 & 666 & 3,44 & 13 & 545 & 2,76 \\
6 & 580 & 3,07 & 14 & 572 & 2,88 \\
7 & 555 & 3,00 & 15 & 594 & 2,96 \\
8 & 661 & 3,43 & & & \\
\end{longtable}

\begin{itemize}
\item
  Façamos \(Y\)=LSAT e \(Z\)=GPA. A estatística (estimador) de interesse
  é o coeficiente de correlação amostral entre as variáveis \(Y\) e
  \(Z\):
  \[\hat{\Theta}=corr(Y,Z)=\dfrac{Cov(Y,Z)}{DP(Y).DP(Z)}=\dfrac{\sum_{i=1}^n(Y_i-\bar{Y})(Z_i-\bar{Z})/n}{DP(Y).DP(Z)}\]
\item
  Para os dados observados, a estimativa do coeficiente de correlação
  amostral é 0.776. Quão acurado é o estimador?
\end{itemize}

\begin{longtable}[]{@{}
  >{\raggedright\arraybackslash}p{(\columnwidth - 16\tabcolsep) * \real{0.2000}}
  >{\raggedright\arraybackslash}p{(\columnwidth - 16\tabcolsep) * \real{0.1000}}
  >{\raggedright\arraybackslash}p{(\columnwidth - 16\tabcolsep) * \real{0.1000}}
  >{\raggedright\arraybackslash}p{(\columnwidth - 16\tabcolsep) * \real{0.1000}}
  >{\raggedright\arraybackslash}p{(\columnwidth - 16\tabcolsep) * \real{0.1000}}
  >{\raggedright\arraybackslash}p{(\columnwidth - 16\tabcolsep) * \real{0.1000}}
  >{\raggedright\arraybackslash}p{(\columnwidth - 16\tabcolsep) * \real{0.1000}}
  >{\raggedright\arraybackslash}p{(\columnwidth - 16\tabcolsep) * \real{0.1000}}
  >{\raggedright\arraybackslash}p{(\columnwidth - 16\tabcolsep) * \real{0.1000}}@{}}
\caption{\textbf{Estimativas bootstrap do erro padrão para
\(\hat{\Theta} = corr(Y,Z)\)}}\tabularnewline
\toprule\noalign{}
\begin{minipage}[b]{\linewidth}\raggedright
\(B\)
\end{minipage} & \begin{minipage}[b]{\linewidth}\raggedright
25
\end{minipage} & \begin{minipage}[b]{\linewidth}\raggedright
50
\end{minipage} & \begin{minipage}[b]{\linewidth}\raggedright
100
\end{minipage} & \begin{minipage}[b]{\linewidth}\raggedright
200
\end{minipage} & \begin{minipage}[b]{\linewidth}\raggedright
400
\end{minipage} & \begin{minipage}[b]{\linewidth}\raggedright
800
\end{minipage} & \begin{minipage}[b]{\linewidth}\raggedright
1600
\end{minipage} & \begin{minipage}[b]{\linewidth}\raggedright
3200
\end{minipage} \\
\midrule\noalign{}
\endfirsthead
\toprule\noalign{}
\begin{minipage}[b]{\linewidth}\raggedright
\(B\)
\end{minipage} & \begin{minipage}[b]{\linewidth}\raggedright
25
\end{minipage} & \begin{minipage}[b]{\linewidth}\raggedright
50
\end{minipage} & \begin{minipage}[b]{\linewidth}\raggedright
100
\end{minipage} & \begin{minipage}[b]{\linewidth}\raggedright
200
\end{minipage} & \begin{minipage}[b]{\linewidth}\raggedright
400
\end{minipage} & \begin{minipage}[b]{\linewidth}\raggedright
800
\end{minipage} & \begin{minipage}[b]{\linewidth}\raggedright
1600
\end{minipage} & \begin{minipage}[b]{\linewidth}\raggedright
3200
\end{minipage} \\
\midrule\noalign{}
\endhead
\bottomrule\noalign{}
\endlastfoot
\(\hat{ep}_B\) & 0,140 & 0,142 & 0,151 & 0,143 & 0,141 & 0,137 & 0,133 &
0,132 \\
\end{longtable}

\begin{tcolorbox}[enhanced jigsaw, coltitle=black, bottomtitle=1mm, toprule=.15mm, arc=.35mm, colframe=quarto-callout-note-color-frame, breakable, opacityback=0, bottomrule=.15mm, rightrule=.15mm, titlerule=0mm, toptitle=1mm, title=\textcolor{quarto-callout-note-color}{\faInfo}\hspace{0.5em}{Nota}, leftrule=.75mm, opacitybacktitle=0.6, left=2mm, colback=white, colbacktitle=quarto-callout-note-color!10!white]

No caso de valores extremos inflacionarem fortemente \(\hat{ep}_B\), uma
medida mais robusta para o estimador bootstrap do erro padrão é
desejável. (ver Efron \(\&\) Tibshirani (1993)).

\end{tcolorbox}

\begin{tcolorbox}[enhanced jigsaw, coltitle=black, bottomtitle=1mm, toprule=.15mm, arc=.35mm, colframe=quarto-callout-note-color-frame, breakable, opacityback=0, bottomrule=.15mm, rightrule=.15mm, titlerule=0mm, toptitle=1mm, title=\textcolor{quarto-callout-note-color}{\faInfo}\hspace{0.5em}{Nota}, leftrule=.75mm, opacitybacktitle=0.6, left=2mm, colback=white, colbacktitle=quarto-callout-note-color!10!white]

Inferências baseadas na distribuição normal são ``questionáveis'' quando
o histograma das réplicas bootstrap indica forte assimetria.

\end{tcolorbox}

\subsubsection{Exercícios:}\label{exercuxedcios-3}

\begin{enumerate}
\def\labelenumi{(\arabic{enumi})}
\item
  Considere \(B = 3200\). Para cada uma das 3200 amostras bootstrap,
  obtenha a réplica bootstrap \(\hat{\theta}^*=corr(y^*,z^*)\). Faça o
  histograma das réplicas.
\item
  A Tabela 3.2, p.~21 do livro texto, apresenta os dados populacionais
  das 82 faculdades. Selecione 3200 amostras aleatórias de tamanho
  \(n = 15\). Para cada uma dessas amostra calcule o coeficiente de
  correlação e faça o histograma.
\end{enumerate}

\subsection{Bootstrap Paramétrico}\label{bootstrap-paramuxe9trico}

O estimador bootstrap paramétrico do erro padrão é definido por
\[ep_{\hat{F}_{par}}(\hat\Theta^*),\] em que \(\hat{F}_{par}\) é um
estimador de \(F\) derivado do modelo paramétrico para os dados.

Para os dados das faculdades: Vamos supor que a população (LSAT, GPA)
possa ser descrita por um modelo paramétrico normal bivariado \(F\).
Estimamos \(F\) por \(\hat{F}_{normal}\), que denota a f.d.a. de uma
normal bivariada com vetor de médias e matriz de covariâncias
\((\bar{y},\bar{z})\) e \(\dfrac{1}{14}\left(
\begin{array}{ll}
\sum(y_i-\bar{y})^2            &  \sum(y_i-\bar{y})(z_i-\bar{z}) \\
\sum(y_i-\bar{y})(z_i-\bar{z}) &  \sum(z_i-\bar{z})^2  \\
\end{array}
\right)\).

O estimador bootstrap paramétrico do erro padrão da correlação
\(\hat{\Theta}\) será dado por \(ep_{\hat{F}_{normal}}(\hat\Theta^*)\).
Esse estimador bootstrap ideal será aproximado por \(\hat{ep}_{B}\)
(conforme algoritmo a seguir).

\textbf{O algoritmo bootstrap:}

\begin{itemize}
\tightlist
\item
  extrair \(B\) amostras de tamanho \(n\) de \(\hat{F}_{par}\):
  \(\underset{\sim}{x}^{*^{1}}, \underset{\sim}{x}^{*^{2}}, \ldots, \underset{\sim}{x}^{*^{B}}\)
\item
  calcular \(s(\underset{\sim}{x}^{*^{b}})\), \(b=1, 2, \ldots, B\).
  \(s(\underset{\sim}{x}^{*^{b}})\) é a réplica bootstrap de
  \(s(\underset{\sim}{x})\)
\item
  calcular
  \(\hat{ep}_{boot}=\hat{ep}_{B}=\sqrt{ \dfrac{\sum_{b=1}^{B}[s(\underset{\sim}{x}^{*^{b}})-s(.)]^2}{B-1}},\)em
  que \(s(.)=\frac{\sum_{b=1}^{B}s(\underset{\sim}{x}^{*^{b}})}{B}\)
\end{itemize}

No exemplo das faculdades, assumindo o modelo normal bivariado,
extraímos \(B\) amostras de tamanho \(n=15\) de \(\hat{F}_{normal}\),
calculamos o coeficiente de correlação para cada amostra e, por fim,
calculamos o desvio padrão desses coeficientes de correlação. Usando
\(B=3200\) encontramos \(\hat{ep}_{B}=0.124\), que é próximo ao valor
0.131 obtido com o bootstrap não-paramétrico.

A fórmula teórica para o erro padrão do coeficiente de correlação é
\(\dfrac{1-\hat{\Theta}^2}{\sqrt{n-3}}\), com
\(\hat{\Theta}=corr(Y,Z)\). Vimos que \(\hat{\theta}=0.776\), o que
resulta a estimativa 0.115 para o erro padrão do coeficiente de
correlação entre as variáveis \(Y\)=LSAT e \(Z\)=GPA.

Transformação de Fisher para o coeficiente de correlação
\(\hat{\Theta}\):
\(\hat{\zeta}=0.5\log\left(\dfrac{1+\hat{\Theta}}{1-\hat{\Theta}}\right)\).
Assim, \(\hat{\zeta}\) tem distribuição aproximadamente normal com média
\(0.5\log\left(\dfrac{1+\Theta}{1-\Theta}\right)\) e variância
\(\dfrac{1}{n-3}\). O erro padrão para \(\hat{\zeta}\) é
\(\sqrt{\dfrac{1}{n-3}}\). Para o exemplo das faculdades, o valor é
\(\dfrac{1}{\sqrt{12}}=0.289\).

A título de comparação com o procedimento bootstrap, a estatística
(estimador) \(\hat{\zeta}\) foi estimado em cada uma das \(B=3200\)
amostras bootstrap. O desvio padrão das réplicas bootstrap resultou
0.290 (muito próximo ao valor teórico 0.289). Histogramas para as
correlações \(\hat{\theta}^*\)e para os \(\hat{\zeta}^*\).

\begin{tcolorbox}[enhanced jigsaw, coltitle=black, bottomtitle=1mm, toprule=.15mm, arc=.35mm, colframe=quarto-callout-important-color-frame, breakable, opacityback=0, bottomrule=.15mm, rightrule=.15mm, titlerule=0mm, toptitle=1mm, title=\textcolor{quarto-callout-important-color}{\faExclamation}\hspace{0.5em}{Importante}, leftrule=.75mm, opacitybacktitle=0.6, left=2mm, colback=white, colbacktitle=quarto-callout-important-color!10!white]

Muitas fórmulas para os erros padrões são aproximações baseadas na
teoria normal e isso ``explica'' os resultados próximos obtidos com o
uso do bootstrap paramétrico que extrai amostras a partir da
distribuição normal.

\end{tcolorbox}

Vantagens do bootstrap sobre os métodos tradicionais:

\begin{itemize}
\tightlist
\item
  \textbf{bootstrap não-paramétrico:} não é necessário fazer suposições
  de modelos paramétricos para a população;
\item
  \textbf{bootstrap paramétrico:} possibilita estimar erros padrões em
  problemas para os quais não há fórmulas para os erros padrões.
\end{itemize}

\section{Jackknife}\label{jackknife}

\subsection{Introdução}\label{introduuxe7uxe3o-3}

Jackknife é uma técnica para estimar viés e erro padrão de estimadores.
É uma técnica que antecede o bootstrap e foi proposta no trabalho
pioneiro Quenouille (1949) para reduzir viés do estimador da correlação
serial.

\textbf{Formalização:}

\begin{itemize}
\item
  \(\underset{\sim}{X} = (X_1, X_2, \dots, X_n)\): amostra aleatória de
  uma f.d.a. \(F\)
\item
  \(\underset{\sim}{x} = (x_1, x_2, \dots, x_n)\): amostra aleatória
  observada de \(F\)
\item
  \(\Theta=t(F)\): parâmetro (\(\Theta\) uma função de \(F\))
\item
  \(\hat{\Theta}=s(\underset{\sim}{X})\): estimador para \(\Theta\)
\item
  \(\hat{\theta}=s(\underset{\sim}{x})\): estimativa para \(\Theta\)
\end{itemize}

O jackknife toma como base \(n\) amostras de tamanho \(n-1\)
selecionadas da amostra aleatória observada. A \(i\)-ésima amostra
jackknife consiste da amostra observada com a \(i\)-ésima observação
removida, \(i=1,2, \ldots, n\):
\[{\underset{\sim}{x}}_{(i)}=(x_1,x_2,\ldots, x_{i-1},x_{i+1}, \ldots, x_n).\]

\subsection{Estimador do viés}\label{estimador-do-viuxe9s}

Para cada amostra \(i\) jackknife, é obtida a réplica jackknife
\(\hat{\theta}_{(i)}=s({\underset{\sim}{x}}_{(i)})\) e o estimador
jackknife do viés de \(\hat{\Theta}\) é dado por:
\[\widehat{\mbox{viés}}_{jack} =(n-1)( \hat{\Theta}_{(\cdot)} - \hat{\Theta} ),\]
em que
\(\hat{\Theta}_{(\cdot)}=\sum_{i=1}^n\dfrac{\hat{\Theta}_{(i)}}{n}\).

\subsection{Estimado do erro padrão}\label{estimado-do-erro-padruxe3o}

O estimador jackknife do erro padrão de \(\hat{\Theta}\) pode ser
escrito da seguinte maneira:
\[\widehat{ep}_{jack} =\left[ \dfrac{n-1}{n}\sum_{i=1}^n (\hat{\Theta}_{(i)} - \hat{\Theta}_{(\cdot)})^2\right]^{\frac{1}{2}},\]
em que
\(\hat{\Theta}_{(\cdot)}=\sum_{i=1}^n\dfrac{\hat{\Theta}_{(i)}}{n}\).

Algumas considerações importantes:

\begin{itemize}
\item
  Bootstrap: amostragem aleatória com reposição;
\item
  Jackknife: amostras fixas;
\item
  Jackknife requer o cálculo do estimador apenas para \(n\) amostras;
\item
  A acurácia do estimador jackknife do erro padrão depende de quão
  próximo o estimador é da linearidade. Para funções fortemente não
  lineares, o jackknife pode ser ineficiente;
\item
  Estimador linear:
  \(\hat{\Theta}=s(\underset{\sim}{X})=\mu +\frac{1}{n}\sum_{i=1}^n\alpha(X_i).\)
\end{itemize}

\subsubsection{Exemplo}\label{exemplo-7}

Considere a amostra observada: \(\underset{\sim}{x}=(10,26,30,40,48)\) e
o estimador: mediana(\(\hat{\Theta}\)). As amostras e réplicas jackknife
são dadas, respectivamente, por:

\begin{enumerate}
\def\labelenumi{\arabic{enumi}.}
\tightlist
\item
  \({\underset{\sim}{x}}_{(1)}=(26,30,40,48)\) e
  \(\hat{\theta}_{(1)}=35;\)
\item
  \({\underset{\sim}{x}}_{(2)}=(10,30,40,48)\) e
  \(\hat{\theta}_{(2)}=35;\)
\item
  \({\underset{\sim}{x}}_{(3)}=(10,26,40,48)\) e
  \(\hat{\theta}_{(3)}=33;\)
\item
  \({\underset{\sim}{x}}_{(4)}=(10,26,30,48)\) e
  \(\hat{\theta}_{(4)}=28;\)
\item
  \({\underset{\sim}{x}}_{(5)}=(10,26,30,40)\) e
  \(\hat{\theta}_{(5)}=28.\)
\end{enumerate}

Desta forma, a estimativa jackknife do erro padrão de \(\hat{\Theta}\):
\[\widehat{ep}_{jack} =\left[ \dfrac{4}{5}\sum_{i=1}^5 (\hat{\theta}_{(i)} - \hat{\theta}_{(\cdot)})^2\right]^{\frac{1}{2}}\approx 6.38,\]
com
\(\hat{\theta}_{(\cdot)}=\sum_{i=1}^5\dfrac{\hat{\theta}_{(i)}}{5}=31.8.\)

A seguir, a sintaxe do \texttt{R} para calcular a estimativa jackknife
do erro padrão para a mediana.

\begin{Shaded}
\begin{Highlighting}[]
\NormalTok{x }\OtherTok{\textless{}{-}} \FunctionTok{c}\NormalTok{(}\DecValTok{10}\NormalTok{,}\DecValTok{26}\NormalTok{,}\DecValTok{30}\NormalTok{,}\DecValTok{40}\NormalTok{,}\DecValTok{48}\NormalTok{)}
\NormalTok{n }\OtherTok{\textless{}{-}} \FunctionTok{length}\NormalTok{(x)}

\NormalTok{est.jack }\OtherTok{\textless{}{-}} \DecValTok{0}

\ControlFlowTok{for}\NormalTok{(i }\ControlFlowTok{in} \DecValTok{1}\SpecialCharTok{:}\NormalTok{n)\{ }
\NormalTok{  est.jack[i] }\OtherTok{\textless{}{-}} \FunctionTok{median}\NormalTok{(x[}\SpecialCharTok{{-}}\NormalTok{i])}
\NormalTok{\}}

\NormalTok{ep.jack }\OtherTok{\textless{}{-}} \FunctionTok{sqrt}\NormalTok{(((n}\DecValTok{{-}1}\NormalTok{)}\SpecialCharTok{\^{}}\DecValTok{2}\SpecialCharTok{/}\NormalTok{n)}\SpecialCharTok{*}\FunctionTok{var}\NormalTok{(est.jack))}
\NormalTok{ep.jack}
\end{Highlighting}
\end{Shaded}

\begin{verbatim}
[1] 6.374951
\end{verbatim}

\section{Intervalos de Confiança}\label{intervalos-de-confianuxe7a}

\begin{tcolorbox}[enhanced jigsaw, coltitle=black, bottomtitle=1mm, toprule=.15mm, arc=.35mm, colframe=quarto-callout-warning-color-frame, breakable, opacityback=0, bottomrule=.15mm, rightrule=.15mm, titlerule=0mm, toptitle=1mm, title=\textcolor{quarto-callout-warning-color}{\faExclamationTriangle}\hspace{0.5em}{Notação}, leftrule=.75mm, opacitybacktitle=0.6, left=2mm, colback=white, colbacktitle=quarto-callout-warning-color!10!white]

\begin{itemize}
\item
  \(\underset{\sim}{X} = (X_1, x_2, \dots, X_n)\): amostra aleatória de
  uma f.d.a. \(F\)
\item
  \(\underset{\sim}{x} = (x_1, x_2, \dots, x_n)\): amostra aleatória
  observada de \(F\)
\item
  \(\Theta=t(F)\): parâmetro (\(\Theta\) uma função de \(F\))
\item
  \(\hat{\Theta}=s(\underset{\sim}{X})\): estimador para \(\Theta\)
\item
  \(\hat{\theta}=s(\underset{\sim}{x})\): estimativa para \(\Theta\)
\end{itemize}

\end{tcolorbox}

\section{Intervalos de Confiança Normal
Padrão}\label{intervalos-de-confianuxe7a-normal-padruxe3o}

Suponha que \(\hat{\Theta}\sim N(\Theta, ep(\hat{\Theta})^2)\).
Portanto, se \(ep(\hat{\Theta})\) é conhecido temos que:

\[Z=\dfrac{\hat{\Theta} - \Theta}{ep(\hat\Theta)} \sim N(0,1),\] e
\[IC_z\left(100(1-\alpha)\%,\Theta\right)=\hat{\Theta} \pm z_{\frac{\alpha}{2}}ep(\hat{\Theta}).\]

\section{Intervalo de Confiança
t-Student}\label{intervalo-de-confianuxe7a-t-student}

Suponha que \(\hat{\Theta}\sim N(\Theta, ep(\hat{\Theta})^2)\) e se
\(ep(\hat{\Theta})\) é desconhecido, portanto

\[Z=\dfrac{\hat{\Theta} - \Theta}{\hat{ep}(\hat\Theta)} \sim \mbox{t-Student}(n-1),\]

e

\[IC_t\left(100(1-\alpha)\%,\Theta\right)=\hat{\Theta} \pm t_{(n-1,\frac{\alpha}{2})}\hat{ep}(\hat{\Theta}).\]

\section{\texorpdfstring{IC
bootstrap-\(t\)}{IC bootstrap-t}}\label{ic-bootstrap-t}

Agora vamos definir

\[Z^{*^b}=\dfrac{ \hat{\Theta}^{*^b} - \hat{\Theta} }{\hat{ep}^{*^b}},\]
em que \(\hat{\Theta}^{*^b}=s(\underset{\sim}{X}^{*^b})\) e
\(\hat{ep}^{*^b}\) são obtidos para cada amostra bootstrap
\(\underset{\sim}{x}^*\). O percentil \(100 \cdot \alpha\) de
\(Z^{*^b}\) é estimado pelo valor \(\hat{t}^{(\alpha)}\), tal que
\[ \#\{Z^{*^b}\leq \hat{t}^{(\alpha)} \}/B=\alpha.\]

Por exemplo, se \(B=100\) e a confiança para o intervalo é de 90\(\%\),
a estimativa \(\hat{t}^{(0.05)}\) será o quinto maior valor de
\(Z^{*^b}\) e a estimativa \(\hat{t}^{(0.95)}\) será o nonagésimo quinto
maior valor de \(Z^{*^b}\).

Este procedimento estima a distribuição de \(Z\) (quantidade pivotal)
diretamente dos dados. Não é necessário a suposição teórica de
normalidade e sua distribuição é a mesma para qualquer \theta. O
intervalo (bilateral) bootstrap-t de confiança
\(100\times(1-2\alpha)\%\) para o parâmetro \(\Theta\), denotado por
\(IC_t^*\left(100(1-2\alpha)\%,\Theta\right)\), é definido por:

\[ \left[\hat{\Theta} - \hat{t}^{(1-\alpha)}\hat{ep}(\hat{\Theta}), \hat{\Theta} - \hat{t}^{(\alpha)}\hat{ep}(\hat{\Theta})\right].\]
Emprega-se estimativas \textit{plug-in} para \(\hat{\Theta}\) e
\(\hat{ep}(\hat{\Theta})\); não sendo possível, o erro padrão é estimado
usando bootstrap ou jackknife.

\begin{tcolorbox}[enhanced jigsaw, coltitle=black, bottomtitle=1mm, toprule=.15mm, arc=.35mm, colframe=quarto-callout-important-color-frame, breakable, opacityback=0, bottomrule=.15mm, rightrule=.15mm, titlerule=0mm, toptitle=1mm, title=\textcolor{quarto-callout-important-color}{\faExclamation}\hspace{0.5em}{Importante}, leftrule=.75mm, opacitybacktitle=0.6, left=2mm, colback=white, colbacktitle=quarto-callout-important-color!10!white]

Se \(B\cdot\alpha\) não resultar um número inteiro?

\begin{enumerate}
\def\labelenumi{\arabic{enumi}.}
\item
  Efron \(\&\) Tibshirani (1993): supondo \(\alpha \leq 0.5\), faça
  \(k=\mbox{Int}[(B+1)\cdot\alpha]\) (o maior inteiro
  \(\leq (B+1)\cdot\alpha\) ) e defina os percentis empíricos de ordem
  \(100\cdot\alpha\) e \(100\cdot(1-\alpha)\), respectivamente, pelo
  \(k\)-ésimo e \((B+1-k)\)-ésimo maiores valores de \(Z^{*^b}\).
\item
  Davidson \(\&\) Hinkley (1997): interpolação dos percentis.
\end{enumerate}

\end{tcolorbox}

\begin{tcolorbox}[enhanced jigsaw, coltitle=black, bottomtitle=1mm, toprule=.15mm, arc=.35mm, colframe=quarto-callout-note-color-frame, breakable, opacityback=0, bottomrule=.15mm, rightrule=.15mm, titlerule=0mm, toptitle=1mm, title=\textcolor{quarto-callout-note-color}{\faInfo}\hspace{0.5em}{Considerações}, leftrule=.75mm, opacitybacktitle=0.6, left=2mm, colback=white, colbacktitle=quarto-callout-note-color!10!white]

\begin{itemize}
\item
  Este intervalo pode ser fortemente influenciado por poucos valores
  discrepantes.
\item
  Em geral, é necessário um valor de \(B\) muito superior a 200.
\item
  Para grandes amostras, o IC bootstrap-t pode estar mais próximo do
  nível de cobertura desejado do que os IC Normal ou t-Student.
\item
  Abaixo, percentis da distribuição \(t\)-Student com 8 gl, normal
  padrão e distribuição bootstrap de \(Z^{*^b}\) (para o grupo controle
  no experimento com os ratos; \(B=1000\)).
\end{itemize}

\end{tcolorbox}

\begin{longtable}[]{@{}lllll@{}}
\caption{Comparação dos percentis de diferentes
distribuições.}\tabularnewline
\toprule\noalign{}
percentil & 5 & 10 & 90 & 95 \\
\midrule\noalign{}
\endfirsthead
\toprule\noalign{}
percentil & 5 & 10 & 90 & 95 \\
\midrule\noalign{}
\endhead
\bottomrule\noalign{}
\endlastfoot
\(t_8\) & -1.86 & -1.40 & 1.40 & 1.86 \\
Normal & -1.65 & -1.28 & 1.28 & 1.65 \\
bootstrap-t & -4.53 & -2.01 & 1.19 & 1.53 \\
\end{longtable}

O IC bootstrap-\(t\) para \(\Theta\) (média do tempo de sobrevivência de
ratos não submetidos ao tratamento (grupo controle)):

\[\left[ 56.22-1.53 \times 13.33, \ 56.22+4.53  \times 13.33\right]=\left[35.82, \ 116.74\right],\]
usado a estimativa \textit{plug-in} para o erro padrão da média.

\begin{tcolorbox}[enhanced jigsaw, coltitle=black, bottomtitle=1mm, toprule=.15mm, arc=.35mm, colframe=quarto-callout-tip-color-frame, breakable, opacityback=0, bottomrule=.15mm, rightrule=.15mm, titlerule=0mm, toptitle=1mm, title=\textcolor{quarto-callout-tip-color}{\faLightbulb}\hspace{0.5em}{Considerações Finais}, leftrule=.75mm, opacitybacktitle=0.6, left=2mm, colback=white, colbacktitle=quarto-callout-tip-color!10!white]

\begin{itemize}
\item
  os valores dos percentis bootstrap-\(t\) podem não ser simétricos em
  relação ao zero.
\item
  aplicável para estatísticas de localização (média, mediana etc).
\item
  Cautela no uso envolvendo pequenas amostras (limites do intervalo fora
  do espaço paramétrico).
\item
  o \(IC^*_t\) não é \emph{transformation-respecting} (ver próxima
  seção)
\end{itemize}

\end{tcolorbox}

\subsection{Intervalos de Confiança bootstrap
percentil}\label{intervalos-de-confianuxe7a-bootstrap-percentil}

O intervalo bootstrap percentil de confiança \(100(1-2\alpha)\%\),
denotado por \(IC_p^*\left(100(1-2\alpha)\%,\Theta \right)\), é definido
por

\[\left[ \hat\Theta^{*(\alpha)}, \ \hat\Theta^{*(1-\alpha)}\right],\] em
que \(\hat\Theta^{*(\alpha)}\) é o percentil de ordem 100\(\alpha\) da
distribuição bootstrap de \(\hat\Theta^*\). Esta definição refere-se a
um número infinito de réplicas.

Em situações práticas, o intervalo é aproximado com base nos resultados
de \(B\) réplicas bootstrap:

\[\left[ \hat\Theta_B^{*(\alpha)}, \ \hat\Theta_B^{*(1-\alpha)}\right],\]
em que \(\hat\Theta_B^{*(\alpha)}\) é dado pelo percentil de ordem
100\(\alpha\) dos valores de \(\hat\theta_B^{*(b)}\), isto é, o
\((B\cdot\alpha)\)-ésimo valor ordenado da lista das \(B\) réplicas de
\(\hat\theta_B^{*(b)}\).

\begin{tcolorbox}[enhanced jigsaw, coltitle=black, bottomtitle=1mm, toprule=.15mm, arc=.35mm, colframe=quarto-callout-note-color-frame, breakable, opacityback=0, bottomrule=.15mm, rightrule=.15mm, titlerule=0mm, toptitle=1mm, title=\textcolor{quarto-callout-note-color}{\faInfo}\hspace{0.5em}{Nota}, leftrule=.75mm, opacitybacktitle=0.6, left=2mm, colback=white, colbacktitle=quarto-callout-note-color!10!white]

Se a distribuição de \(\hat\Theta^{*}\) é aproximadamente normal, os
intervalos percentil e normal padrão são próximos.

\end{tcolorbox}

\subsubsection{Exemplo}\label{exemplo-8}

Para uma amostra: \va , \(n=10\), de uma distribuição (população) normal
padrão, considere o parâmetro de interesse \(\Theta\).
\(\Theta=\exp(\mu)\), sendo \(\mu=0\). Sendo a estimativa para
\(\Theta\): \(\hat{\theta}=\exp(\bar{x})\). Além disso, use o bootstrap
não paramétrico: \(B=1000\) réplicas \(\hat{\theta}^*\) usando a amostra
observada: (1.669, -0.411, -0.322, 0.746, -0.868, -0.874, 1.011, -0.173,
0.021, 1.482). Resultando na estimativa para \(\Theta\):
\(\hat{\theta}=\exp(\bar{x})=1.256\).

\begin{longtable}[]{@{}ccccccccc@{}}
\caption{Percentis de \(\hat\theta^*\) com base em 1000 réplicas
bootstrap.}\tabularnewline
\toprule\noalign{}
2.5\% & 5\% & 10\% & 16\% & 50\% & 84\% & 90\% & 95\% & 97.5\% \\
\midrule\noalign{}
\endfirsthead
\toprule\noalign{}
2.5\% & 5\% & 10\% & 16\% & 50\% & 84\% & 90\% & 95\% & 97.5\% \\
\midrule\noalign{}
\endhead
\bottomrule\noalign{}
\endlastfoot
0.74 & 0.81 & 0.89 & 0.95 & 1.25 & 1.65 & 1.77 & 2.00 & 2.13 \\
\end{longtable}

Desta maneira, temos que

\[IC_p^*\left(95\%,\Theta\right)=\left[0.74, 2.13\right],\]

e

\[IC_p^*\left(90\%,\Theta\right)=\left[0.81, 2.00\right].\]\\
Abaixo um código \texttt{R} para o exemplo.

\begin{Shaded}
\begin{Highlighting}[]
\FunctionTok{set.seed}\NormalTok{(}\DecValTok{1234}\NormalTok{)}
\NormalTok{n }\OtherTok{\textless{}{-}} \DecValTok{10}
\NormalTok{x }\OtherTok{\textless{}{-}} \FunctionTok{rnorm}\NormalTok{(n) }\CommentTok{\# amostra observada}
\NormalTok{B }\OtherTok{\textless{}{-}} \DecValTok{1000}     \CommentTok{\# no. de réplicas bootstrap}
\NormalTok{theta }\OtherTok{\textless{}{-}} \DecValTok{0}
\NormalTok{alfa }\OtherTok{\textless{}{-}} \FunctionTok{c}\NormalTok{(}\FloatTok{0.025}\NormalTok{,}\FloatTok{0.05}\NormalTok{,}\FloatTok{0.10}\NormalTok{,}\FloatTok{0.16}\NormalTok{,}\FloatTok{0.50}\NormalTok{,}\FloatTok{0.84}\NormalTok{,}\FloatTok{0.90}\NormalTok{,}\FloatTok{0.95}\NormalTok{,}\FloatTok{0.975}\NormalTok{)}

\ControlFlowTok{for}\NormalTok{(i }\ControlFlowTok{in} \DecValTok{1}\SpecialCharTok{:}\NormalTok{B)\{}
\NormalTok{  a }\OtherTok{\textless{}{-}} \FunctionTok{sample}\NormalTok{(x,n,}\AttributeTok{replace=}\ConstantTok{TRUE}\NormalTok{)  }\CommentTok{\# amostra bootstrap}
\NormalTok{  theta[i] }\OtherTok{\textless{}{-}} \FunctionTok{exp}\NormalTok{(}\FunctionTok{mean}\NormalTok{(a))       }\CommentTok{\# réplica bootstrap }
\NormalTok{\}}

\FunctionTok{exp}\NormalTok{(}\FunctionTok{mean}\NormalTok{(x)) }\CommentTok{\# estimativa}
\end{Highlighting}
\end{Shaded}

\begin{verbatim}
[1] 0.6817056
\end{verbatim}

\begin{Shaded}
\begin{Highlighting}[]
\NormalTok{sz }\OtherTok{\textless{}{-}} \FunctionTok{sort}\NormalTok{(theta)}
\FunctionTok{rbind}\NormalTok{(}\DecValTok{100}\SpecialCharTok{*}\NormalTok{alfa, }\FunctionTok{round}\NormalTok{(sz[}\FunctionTok{c}\NormalTok{(B}\SpecialCharTok{*}\NormalTok{alfa)],}\DecValTok{2}\NormalTok{)) }
\end{Highlighting}
\end{Shaded}

\begin{verbatim}
     [,1] [,2]  [,3]  [,4]  [,5]  [,6] [,7] [,8] [,9]
[1,] 2.50 5.00 10.00 16.00 50.00 84.00   90 95.0 97.5
[2,] 0.37 0.42  0.48  0.52  0.71  0.93    1  1.1  1.2
\end{verbatim}

Agora iremos considerar outra abordagem. Seja \(\Phi=\log(\Theta)\) e
\(\hat\Phi=\log(\hat\Theta)=\bar{X}\). Usando
\(IC_z\left(95\%,\Phi \right)= \hat{\Phi} \pm z_{\frac{\alpha}{2}}\hat{ep}(\hat{\Phi})\),
resulta
\(\left[ 0.228 \pm 1.96 \times 0.28 \right]=\left[-0.32, 0.78\right]\),
em que \(\hat{ep}(\hat{\Phi})\) é o estimador \textit{plug-in} para o
erro padrão de \(\hat\Phi\). Fazendo a transformação inversa, o
intervalo para \(\Theta\) é \(\left[0.73, 2.18\right]\). Note que o
intervalo percentil para \(\Theta\) assemelha-se ao intervalo normal
padrão construído para uma transformação apropriada de \(\Theta\) e
transformado para a escala de \(\Theta\).

\begin{tcolorbox}[enhanced jigsaw, coltitle=black, bottomtitle=1mm, toprule=.15mm, arc=.35mm, colframe=quarto-callout-important-color-frame, breakable, opacityback=0, bottomrule=.15mm, rightrule=.15mm, titlerule=0mm, toptitle=1mm, title=\textcolor{quarto-callout-important-color}{\faExclamation}\hspace{0.5em}{Importante}, leftrule=.75mm, opacitybacktitle=0.6, left=2mm, colback=white, colbacktitle=quarto-callout-important-color!10!white]

\textbf{Vantagem do método percentil:} Não é necessário conhecer a
transformação (normalizadora). O método percentil incorpora
automaticamente a transformação adequada.

\end{tcolorbox}

\begin{tcolorbox}[enhanced jigsaw, coltitle=black, bottomtitle=1mm, toprule=.15mm, arc=.35mm, colframe=quarto-callout-caution-color-frame, breakable, opacityback=0, bottomrule=.15mm, rightrule=.15mm, titlerule=0mm, toptitle=1mm, title=\textcolor{quarto-callout-caution-color}{\faFire}\hspace{0.5em}{Lema}, leftrule=.75mm, opacitybacktitle=0.6, left=2mm, colback=white, colbacktitle=quarto-callout-caution-color!10!white]

Suponha a transformação \(\hat\Phi=m(\hat\Theta)\) e, consequentemente,
\(\hat\Phi\sim N(\Phi,c^2)\), para algum erro padrão \(c\). Então, o
intervalo percentil para \(\Theta\) será dado por
\[\left[ m^{-1}(\hat\Phi -z_{\frac{\alpha}{2}}.c), m^{-1}(\hat\Phi +z_{\frac{\alpha}{2}}.c)\right].\]

\end{tcolorbox}

\begin{tcolorbox}[enhanced jigsaw, coltitle=black, bottomtitle=1mm, toprule=.15mm, arc=.35mm, colframe=quarto-callout-note-color-frame, breakable, opacityback=0, bottomrule=.15mm, rightrule=.15mm, titlerule=0mm, toptitle=1mm, title=\textcolor{quarto-callout-note-color}{\faInfo}\hspace{0.5em}{Nota}, leftrule=.75mm, opacitybacktitle=0.6, left=2mm, colback=white, colbacktitle=quarto-callout-note-color!10!white]

\begin{itemize}
\item
  Em situações nas quais o uso de IC padrão é adequado\(^*\), o IC
  percentil produz resultados próximos ao do IC padrão.
\item
  Em situações nas quais o uso do IC padrão é adequado apenas após uma
  transformação no parâmetro, a aplicaçao do método percentil
  automaticamente contempla essa transformação.
\end{itemize}

\end{tcolorbox}

\begin{tcolorbox}[enhanced jigsaw, coltitle=black, bottomtitle=1mm, toprule=.15mm, arc=.35mm, colframe=quarto-callout-tip-color-frame, breakable, opacityback=0, bottomrule=.15mm, rightrule=.15mm, titlerule=0mm, toptitle=1mm, title=\textcolor{quarto-callout-tip-color}{\faLightbulb}\hspace{0.5em}{Considerações Finais}, leftrule=.75mm, opacitybacktitle=0.6, left=2mm, colback=white, colbacktitle=quarto-callout-tip-color!10!white]

\begin{itemize}
\item
  Propriedade \textit{transformation-respecting}: o intervalo percentil
  para qualquer transformação (monotônica) \(\Phi=m(\Theta)\) do
  parâmetro \(\Theta\) é dado por
  \[\left[ m(\hat\Theta^{*(\alpha)}), \ m(\hat\Theta^{*(1-\alpha)}\right].\]
\item
  A propriedade também é válida para o intervalo aproximado com base nos
  resultados de \(B\) réplicas bootstrap:
  \[\left[ m(\hat\Theta_B^{*(\alpha)}), \ m(\hat\Theta_B^{*(1-\alpha)})\right].\]
\item
  Propriedade \emph{range-preserving}: produz intervalos com limites
  dentro do espaço paramétrico.
\end{itemize}

\end{tcolorbox}

\subsection{Intervalos de Confiança bootstrap - versões
aprimoradas}\label{intervalos-de-confianuxe7a-bootstrap---versuxf5es-aprimoradas}

O intervalo bootstrap-\(t\) apresenta boa probabilidade de cobertura
teórica, mas tende a ser irregular na prática. Já o intervalo bootstrap
percentil é menos irregular, mas apresenta probabilidade de cobertura
menos satisfatória.

\textbf{Proposta:} intervalo bootstrap \(BC_a\) (\emph{bias-corrected
and accelerated}) e \(ABC\) (\emph{approximate bootstrap confidence}).

substancial melhoria na prática e na teoria. correção de viés do
estimador. o \(ABC\) exige menos esforço computacional do que o \(BC_a\)

\bookmarksetup{startatroot}

\chapter{Métodos de Monte Carlo}\label{muxe9todos-de-monte-carlo}

\section{Introdução}\label{introduuxe7uxe3o-4}

\section{Integração de Monte
Carlo}\label{integrauxe7uxe3o-de-monte-carlo}

\section{Erro de Monte Carlo}\label{erro-de-monte-carlo}

\section{Monte Carlo via Função de
Importância}\label{monte-carlo-via-funuxe7uxe3o-de-importuxe2ncia}

\section{Método de Máxima
Verossimilhança}\label{muxe9todo-de-muxe1xima-verossimilhanuxe7a}

\bookmarksetup{startatroot}

\chapter{Algoritmo EM}\label{algoritmo-em}

\bookmarksetup{startatroot}

\chapter{Métodos Adicionais}\label{muxe9todos-adicionais}

\bookmarksetup{startatroot}

\chapter*{References}\label{references}
\addcontentsline{toc}{chapter}{References}

\markboth{References}{References}

\phantomsection\label{refs}
\begin{CSLReferences}{1}{0}
\bibitem[\citeproctext]{ref-falk_closer_2014}
Falk, Ruma. 2014. {``A Closer Look at the Notorious Birthday
Coincidences''}. \emph{Teaching Statistics} 36 (2): 41--46.
\url{https://doi.org/10.1111/test.12014}.

\bibitem[\citeproctext]{ref-hodgson_simulation_2000}
Hodgson, Ted, e Maurice Burke. 2000. {``On Simulation and the Teaching
of Statistics''}. \emph{Teaching Statistics} 22 (3): 91--96.
\url{https://doi.org/10.1111/1467-9639.00033}.

\bibitem[\citeproctext]{ref-martins_learning_2018}
Martins, Rui Manuel Da Costa. 2018. {``Learning the Principles of
Simulation Using the Birthday Problem''}. \emph{Teaching Statistics} 40
(3): 108--11. \url{https://doi.org/10.1111/test.12164}.

\bibitem[\citeproctext]{ref-matthews_coincidences_1998}
Matthews, Robert, e Fiona Stones. 1998. {``Coincidences: the truth is
out there''}. \emph{Teaching Statistics} 20 (1): 17--19.
https://doi.org/\url{https://doi.org/10.1111/j.1467-9639.1998.tb00752.x}.

\bibitem[\citeproctext]{ref-thomas_cusum_1980}
Thomas, F. H., e J. L. Moore. 1980. {``{CUSUM}: Computer Simulation for
Statistics Teaching''}. \emph{Teaching Statistics} 2 (1): 23--28.
\url{https://doi.org/10.1111/j.1467-9639.1980.tb00374.x}.

\bibitem[\citeproctext]{ref-tintle_combating_2015}
Tintle, Nathan, Beth Chance, George Cobb, Soma Roy, Todd Swanson, e Jill
VanderStoep. 2015. {``Combating Anti-Statistical Thinking Using
Simulation-Based Methods Throughout the Undergraduate Curriculum''}.
\emph{The American Statistician} 69 (4): 362--70.
\url{https://doi.org/10.1080/00031305.2015.1081619}.

\bibitem[\citeproctext]{ref-ZiefflerGarfield2007}
Zieffler, Andrew, e Joan B. Garfield. 2007. {``Studying the Role of
Simulation in Developing Students' Statistical Reasoning''}. Em
\emph{Proceedings of the 56th Session of the International Statistical
Institute (ISI)}. International Statistical Institute.

\end{CSLReferences}



\end{document}
